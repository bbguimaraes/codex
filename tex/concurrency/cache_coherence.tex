\section{Cache coherence}

\label{sec:conc:coherence}

This section combines the material in sections \secref{subsec:arch:cache} and
\secref{sec:conc:consistency} to describe the protocols used by CPUs to keep all
memory caches synchronized.  Protocols used in modern CPUs are complex, but the
basic \textit{MESI} protocol can be used to describe the most fundamental
aspects.  It is named after its four states:

\begin{description}
    \item[Modified]
        lines are present only in this CPU's cache and have had their value
        changed.  These lines are owned by the CPU and it must either write it
        back to main memory or transfer ownership to another cache at some
        point.
    \item[Exclusive]
        lines are present only in this CPU's cache and contain the same value as
        the one in memory, but the CPU plans to modify its value in the near
        future (e.g. it is in the middle of a read-modify-write operation).
        These lines are also owned by the CPU but can be discarded at any time.
    \item[Shared]
        lines are present in this CPU's cache and potentially in others, and
        have not been modified.  If it needs to be modified, the other CPUs have
        to be consulted.
    \item[Invalid]
        lines are not present in the cache and are immediate targets for
        replacement by new lines added to the cache.
\end{description}

Each line in the cache has this two-bit state together with the address
information and data.  CPUs transition lines between states when necessary to
effect operations by exchanging messages with other CPUs.  Multicore processors
are message-passing systems at their core.  In systems where all CPUs are
interconnected, the messages are:

\begin{description}
    \item[Read] requests the contents of a memory address.
    \item[Read Response]
        contains the data requested by a \emph{read} message, supplied either by
        the memory system or by a CPU which holds the line in the
        \emph{modified} state.
    \item[Invalidate] requests that a given address be removed from the caches.
    \item[Invalidate Acknowledge] responds to an \emph{invalidate} message.
    \item[Read Invalidate]
        a combination of the \emph{read} and \emph{invalidate} messages.
        Requires a \emph{read response} and a set of \emph{invalidate
        acknowledge} responses.
    \item[Writeback]
        writes data from a \emph{modified} line to a memory address.
\end{description}

Lines can transition from one of the states to any of the other three, depending
on the operations being performed by the CPUs:

\begin{description}
    \item[M $\to$ E]
        a CPU issues a \emph{writeback} message to write the contents of a line
        to memory while retaining ownership of it.
    \item[E $\to$ M]
        an already-owned line is modified, no messages are required.
    \item[M $\to$ I]
        a CPU receives a \emph{read invalidate} message for one of its
        \emph{modified} lines.  It removes the line from its cache and sends
        both a \emph{read response} message with the data and an
        \emph{invalidate acknowledge} message.
    \item[I $\to$ M]
        a CPU issues a RMW operation on a line that is not in its cache.  It
        sends a \emph{read invalidate} message and receives the data in a
        \emph{read response} message, along with a set of \emph{invalidate
        acknowledge} messages, at which point it can complete the transition.
    \item[S $\to$ M]
        a CPU issues a RMW operation on a line that is in its cache.  It sends
        an \emph{invalidate} message and receives a set of \emph{invalidate
        acknowledge} messages, at which point it can complete the transition.
    \item[M $\to$ S]
        a CPU receives a \emph{read} message for one of its \emph{modified}
        lines.  It sends a \emph{read response} with the data and possibly
        writes the line back to memory.
    \item[E $\to$ S]
        a CPU receives a \emph{read} message for one of its \emph{modified}
        lines.  It sends a \emph{read response} with the data.
    \item[S $\to$ E]
        a CPU will soon modify a line that is in its cache.  It sends an
        \emph{invalidate} message and receives a set of \emph{invalidate
        acknowledge} messages, at which point it can complete the transition.
    \item[E $\to$ I]
        a CPU receives a \emph{read invalidate} message for one of its
        \emph{exclusive} lines.  It removes the line from its cache and sends
        both a \emph{read response} message with the data and an
        \emph{invalidate acknowledge} message.
    \item[I $\to$ E]
        a CPU issues a store to a line not in its cache.  It sends a
        \emph{read invalidate} message and receives the data in a \emph{read
        response} message, along with a set of \emph{invalidate acknowledge}
        messages, at which point it completes the store and transitions to
        \emph{modified}.
    \item[I $\to$ S]
        a CPU loads a line not in its cache.  It sends a \emph{read} message and
        receives a \emph{read response}.
    \item[S $\to$ I]
        a CPU receives an \emph{invalidate} message from another CPU doing a
        store to a \emph{shared} line.  It responds with an \emph{invalidate
        acknowledge}.
\end{description}
