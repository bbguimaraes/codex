\section{Fundamentals}

\subsection{Intervals and ranges}
\label{subsec:algo:ranges}

\subsection{Utilities}

Throughout this chapter, a number of utility functions will be used in the
implementation of algorithms.  Some will be described as part of each section,
but a few basic ones are presented here.

\subsubsection{\texttt{check\_range}}

Verifies that an iterator/sentinel pair represents a valid range.  It is meant
as a debugging aid and only used in assertions.  The specific test depends on
the types of the inputs:

\begin{itemize}
    \item
        Contiguous and random-access iterators are tested using
        \texttt{std::distance}, a constant-time operation for those types.
    \item
        A forward iterator needs to be advanced until the sentinel is reached,
        also done using \texttt{std::distance}.  This is an invalid operation
        for invalid ranges, but so will be their use in the algorithms where
        \texttt{check\_range} is used.  For this reason, the actual value of the
        distance does not matter.
    \item
        Input iterators cannot be checked, since they only yield their value
        once.
\end{itemize}

These cases can be nicely discriminated using constraints and concepts from the
standard library.

\lstinputlisting[style=c++,linerange=4-14]{algo/fundamentals/utils.hpp}
\vspace{-\baselineskip}

\subsubsection{\texttt{contains}}

Uses \texttt{check\_range} to determine whether an iterator points to a valid
element of a range.

\lstinputlisting[style=c++,linerange=16-19]{algo/fundamentals/utils.hpp}
\vspace{-\baselineskip}

\subsubsection{\texttt{is\_sorted}}

A C version of the same function in the C++ standard library.

\lstinputlisting[style=c,firstline=3,lastline=10]{algo/fundamentals/utils.h}
\vspace{-\baselineskip}

\subsubsection{\texttt{is\_min\_element}}

Verifies that a given value is the minimum element of a range of values --- i.e.
that no other element is less than it.  It is conceptually equivalent to
\texttt{x <= *std::min\_element(b, e)}, but uses the less-than operator and
stops at the first element for which the assertion does not hold.

\lstinputlisting[style=c++,linerange=21-24]{algo/fundamentals/utils.hpp}
\vspace{-\baselineskip}

\subsubsection{\texttt{min\_element}}

A reimplementation of \texttt{std::min\_element} for illustrative purposes.

\lstinputlisting[style=c++,linerange=26-40]{algo/fundamentals/utils.hpp}
\vspace{-\baselineskip}

\subsubsection{\texttt{lsearch}}

Used in contrast to \texttt{bsearch} in examples in C.

\lstinputlisting
    [style=c,firstline=12,lastline=17]
    {algo/fundamentals/utils.h}
\vspace{-\baselineskip}
