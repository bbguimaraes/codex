\section{Linked lists}

Linked lists are a simple data structure which consist of a sequence of elements
and the connections between them (\textit{links}).  This contrasts with arrays,
which contain just the elements themselves, where the order is implicit in the
fact that elements are laid out sequentially in memory\footnotemark.  Even so,
the underlying storage for the elements is immaterial: the definition of the
data structure is only concerned with how elements are connected to each other.

\footnotetext{
    Subject to alignment and padding requirements, but still extrinsic and
    independent of the contents of each element.}

\subsection{Singly linked list}

The simplest possible implementation of a linked list is shown in figure
\ref{fig:struct:array_list0}.  A fixed-size array is used to store the elements
(separated by solid lines), where each is some data and the index of the next
element (denoted with arrows).  The iteration process starts at the list
\textit{head}, denoted by the arrow starting on the left.  After an element is
processed, the next element is obtained by adding the element's ``next'' index
to the list head.  A special index value denotes the last element, the list
\textit{tail}.  Iteration stops after this element is processed.

\begin{figure}[ht]
    \centering
    \begin{tikzpicture}[
        rectangle/.style={minimum height = 2em},
        rectangle split draw splits = false,
    ]
        \matrix {
            \node (n0) [lnode] {~ \nodepart{two} 1}; &
            \node (n1) [lnode] {~ \nodepart{two} 2}; &
            \node (n2) [lnode] {~ \nodepart{two} 3}; &
            \node (n3) [lnode] {~ \nodepart{two} 4}; &
            \node (n4) [lnode] {~ \nodepart{two} 5}; &
            \node (n5) [lnode] {~ \nodepart{two} 6}; &
            \node (n6) [lnode] {~ \nodepart{two} 7}; &
            \node (n7) [lnode] {\phantom{7}}; \\
            \node {0}; & \node {1}; & \node {2}; & \node {3}; &
            \node {4}; & \node {5}; & \node {6}; & \node {7}; \\
        };
        \foreach \n in {n0, n1, n2, n3, n4, n5, n6, n7} {
            \draw[dashed]
                ($ (\n.north west)!0.5!(\n.north east) $)
                -- ($ (\n.south west)!0.5!(\n.south east) $);
        }
        \draw ($ (n0) - (1, 0) $) edge[->] (n0.west);
        \foreach \n/\m in {n0/n1, n1/n2, n2/n3, n3/n4, n4/n5, n5/n6, n6/n7} {
            \draw
                ($ (\n.north west)!0.75!(\n.north east) $)
                -- +(0, 0.25)
                -- ($ (\m.north west)!0.5!(\m.north east) + (0, 0.25) $)
                edge[->] +(0, -0.25);
        }
    \end{tikzpicture}
    \caption{Array as linked list}
    \label{fig:struct:array_list0}
\end{figure}

Because there is a single link in each element that points forward to the next
element, this type of structure is called a \textit{singly linked list} or
\textit{forward list}.  One advantage they have over arrays becomes apparent
when elements are added and/or removed.  When these operations are performed in
an array in any position that is not the last, elements must be shift either up
or down to make or fill the space for the target position.  In a linked list,
these operations simply manipulate the links between elements.

Figure \ref{fig:struct:array_list1} shows the same list after a few of these
operations.  The sequence of elements is now \texttt{0, 2, 4, 1, 5, 6} even
though they all retain their previous location in memory.

\begin{figure}[ht]
    \centering
    \begin{tikzpicture}[
        rectangle/.style={minimum height = 2em},
        rectangle split draw splits = false,
    ]
        \matrix {
            \node (n0) [lnode] {~ \nodepart{two} 2}; &
            \node (n1) [lnode] {~ \nodepart{two} 6}; &
            \node (n2) [lnode] {~ \nodepart{two} 4}; &
            \node (n3) [lnode] {\phantom{3}}; &
            \node (n4) [lnode] {~ \nodepart{two} 1}; &
            \node (n5) [lnode] {~ \nodepart{two} 6}; &
            \node (n6) [lnode] {~ \nodepart{two} \phantom{7}}; &
            \node (n7) [lnode] {\phantom{7}}; \\
            \node {0}; & \node {1}; & \node {2}; & \node {3}; &
            \node {4}; & \node {5}; & \node {6}; & \node {7}; \\
        };
        \foreach \n in {n0, n1, n2, n3, n4, n5, n6, n7} {
            \draw[dashed]
                ($ (\n.north west)!0.5!(\n.north east) $)
                -- ($ (\n.south west)!0.5!(\n.south east) $);
        }
        \draw ($ (n0) - (1, 0) $) edge[->] (n0.west);
        \foreach \n/\m in {n0/n2, n2/n4, n5/n6} {
            \draw
                ($ (\n.north west)!0.75!(\n.north east) $)
                -- +(0, 0.25)
                -- ($ (\m.north west)!0.5!(\m.north east) + (0, 0.25) $)
                edge[->] +(0, -0.25);
        }
        \draw
            ($ (n4.north west)!0.75!(n4.north east) $)
            -- +(0, 0.5)
            -- ($ (n1.north west)!0.5!(n1.north east) + (0, 0.5) $)
            edge[->] +(0, -0.5);
        \draw
            ($ (n1.north west)!0.75!(n1.north east) $)
            -- +(0, 0.75)
            -- ($ (n5.north west)!0.5!(n5.north east) + (0, 0.75) $)
            edge[->] +(0, -0.75);
    \end{tikzpicture}
    \caption{Array as linked list}
    \label{fig:struct:array_list1}
\end{figure}

Using an index as the link from one element to the next has some advantages.
The array where elements are stored can be freely moved from one memory location
to another, or even serialized/stored/de-serialized and the list will still
retain its proper order.  Indices can also be made as small as desired/possible.
However, a much more common design is to use a \emph{pointer} to the next
element, with a null pointer representing the end of the list.  Figure
\ref{fig:struct:list} shows the classical singly list implementation, where
element indices are for clarity and do not represent relative or absolute
location in memory.

\begin{figure}[ht]
    \centering
    \begin{tikzpicture}[rectangle/.style={minimum height = 2em}]
        \matrix[column sep = 1.5em] {
            \node (n0) [lnode] {\nodepart{two} \&0}; &
            \node (n1) [lnode] {\nodepart{two} \&1}; &
            \node (n2) [lnode] {\nodepart{two} \&2}; &
            \node (n3) [lnode] {\nodepart{two} \&3}; &
            \node (n4) [lnode] {\nodepart{two} \&4}; &
            \node (n5) [lnode] {\nodepart{two} \&5}; &
            \node (n6) [lnode] {\nodepart{two} \&6}; &
            \node (n7) [lnode] {\nodepart{two} 0}; \\
            \node {0}; & \node {1}; & \node {2}; & \node {3}; &
            \node {4}; & \node {5}; & \node {6}; & \node {7}; \\
        };
        \draw ($ (n0) - (1, 0) $) edge[->] (n0.west);
        \draw (n0.east) edge[->] (n1.west);
        \draw (n1.east) edge[->] (n2.west);
        \draw (n2.east) edge[->] (n3.west);
        \draw (n3.east) edge[->] (n4.west);
        \draw (n4.east) edge[->] (n5.west);
        \draw (n5.east) edge[->] (n6.west);
        \draw (n6.east) edge[->] (n7.west);
    \end{tikzpicture}
    \caption{Linked list with pointers}
    \label{fig:struct:list}
\end{figure}

Using pointers further dissociates the implementation of the list from the
underlying storage, as elements still can but are no longer required to be
contiguous and can instead be placed anywhere in memory.  In this type of
implementation, list elements are often referred to as \textit{nodes}.  Several
allocation strategies can be used depending on the context, ranging from
dynamically allocating every element individually, to using blocks or pools, to
using a single array as above.  These all have different advantages and
disadvantages in terms of number of allocations, total memory usage, asymptotic
complexity of operations, iterator invalidation, memory coherence, etc.

Since links are just pointers and can be dereferenced directly, it is no longer
required to keep the list head during iteration.  In fact, iteration can start
at any element of the list given just its memory address.  This makes linked
lists a good candidate for storage in a library that exposes pointers to
(sub-)objects, as the list element can remain in place when others are
added/removed and can be reached easily from the pointer given to the client.
Furthermore, the list head and the next pointer of each element are now the same
type of value, which simplifies the implementation of many list operations, as
we will see.

The simplest way to guarantee a list node can be reached from a pointer to the
value it is contained in is to make the value ``inherit from'' it\footnotemark,
either directly or via an intermediary structure.  This works, since a pointer
to an object of a given type can always be converted to a pointer to an object
of a type it is a strict subset of, but has several disavantages.  It forces the
node pointer to be the first element of the value structure, which imposes a
restriction on how fields can be ordered to improve structure layout and cache
utilization.  It also prevents a value from being placed in more than one list
at a time.

\footnotetext{
    V. section \secref{sec:c++:oop} for a detailed analysis of the patterns
    described in the next paragraphs.}

The alternative, used heavily in the Linux kernel (\cite{Brown2009}), is to
place the list node structure anywhere in the contained value and use the
\texttt{container\_of} macro function to offset the node pointer and reach the
value.  Since both the list head and the nodes are of type \texttt{struct
node*}, they can be treated homogeneously to implement several list operations
(listing \ref{lst:struct:list_ops}).

\lstinputlisting[
    style=c,
    caption={Linked list operations},
    label={lst:struct:list_ops},
    firstline=6,
]{struct/list/ops.c}

\subsubsection{Performance}

The flexibility to easily store and manipulate lists comes at a cost.  As stated
in section \secref{subsec:arch:cache}, modern processors are optimized for
sequential, predictable memory loads.  Contiguous (or mostly contiguous) data
structures such as arrays can be processed very quickly, for two reasons.
First, more than one element can be loaded from main memory or data caches at
once if they are in the same cache line.  Second, since the address of the next
element is always known in advance: each element is at a fixed offset from the
previous: an optimizing compiler or super-scalar processor can issue the memory
loads for an element before the current one is processed, reducing the latency
when processing each element and improving overall throughput.

In the case of a list, the address of the next element is not known until the
current one is examined.  This is a direct data dependency, which means fetching
and processing elements has to be a serial operation.  Since reordering elements
merely manipulates the links and does not reorder them in memory, iterating over
a list can quickly turn into random memory accesses.  Certain types of list
storage also have the potential to spread elements throughout memory with little
coherency.  For all these reasons, even though theoretically the iteration over
an array and a list have the same time complexity ($\text{O}(n)$), the former is
likely to be significantly faster in a modern processor.

One mitigation in certain cases is to place an explicit software prefetch
instruction in the iteration code.  This can in some cases mitigate the latency
of the memory load and result in better interleaving of iteration and processing
code\footnotemark.

\footnotetext{
    V. commit \texttt{75d65a425c0163d3ec476ddc12b51087217a070c} in the Linux
    kernel, which removed this type of software prefetching in the common list
    functions, for subtle cases where this can result in \emph{slower} code.}

\begin{lstlisting}
for(struct node *n = l; prefetch(n->next), n; n = n->next)
    // ...
\end{lstlisting}

\subsection{Doubly linked list}

\subsection{Exercises}

\begin{enumerate}
    \item
        \label{ex:struct:list_reverse}
        Given the definition of a list below, implement the function
        \texttt{list\_reverse}, which is given a list and reverses the order of
        the nodes in place, returning the head of the new list.
        \begin{lstlisting}[style=c]
struct node {
    struct node *next;
    int value;
};

struct node *list_reverse(struct node *n);
        \end{lstlisting}
    \item
        \label{ex:struct:list_remove}
        Given the definition of a list in exercise \ref{ex:struct:list_reverse},
        implement the function \texttt{list\_remove}, which is given a list and
        a value and removes from the list the first node that contains that
        value, returning a pointer to the removed node or \texttt{NULL} if the
        value is not found.
        \begin{lstlisting}[style=c]
struct node *list_remove(struct node **l, int value);
        \end{lstlisting}
\end{enumerate}
