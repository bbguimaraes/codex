\section{Fundamentals}

\subsection{Intervals and ranges}
\label{subsec:algo:ranges}

Both in mathematics and computer science, \textit{ranges} (or
\textit{intervals}) of elements are an important concept.  The interval
corresponding to a sequence of integers such as the set $\{1, 2, 3, 4\}$ can be
represented using a few different notations:

\begin{enumerate}
    \item[0.]
        $[1, 4]$, or $\{x | 1 \leq x \leq 4\}$, the integers from $1$ to $4$,
        both inclusive
    \item
        $[1, 5)$, or $\{x | 1 \leq x < 5\}$, the integers from $1$ (inclusive)
        to $5$ (exclusive)
    \item
        $(0, 4]$, or $\{x | 0 < x \leq 4\}$, the integers from $0$ (exclusive)
        to $4$ (inclusive)
    \item
        $(0, 5)$, or $\{x | 0 < x < 5\}$, the integers from $0$ to $5$, both
        exclusive
\end{enumerate}

These intervals can be classified into the following categories:

\begin{description}
    \item[Closed intervals]
        have determinate endpoints (notation 0, whose endpoints are 1 and 4).
    \item[Open intervals]
        have no determinate endpoints, only bounds which can be infinitely
        approached (notation 3, if applied to the set of real numbers,
        $\mathbb{R}$).
    \item[Half-open intervals]
        have both open and closed endpoints (notations 1 and 2).
    \item[Left-closed intervals] have a minimum element (notations 0 and 1).
    \item[Left-open intervals] do not (notations 2 and 3).
    \item[Right-closed intervals] have a maximum element (notations 0 and 2).
    \item[Right-open intervals] do not (notations 1 and 3).
\end{description}

All of the notation examples above are different ways of representing the same
set of integers.  One of them, however, is generally preferred to denote ranges
in computer programs.  The fundamental reason for this is the same that
underlies the usage of zero-based indices in most programming
languages\footnotemark.

\footnotetext{
    Lua being a notable exception where the choice was made deliberately to make
    the language more approachable to non-technical users (the intended primary
    audience), ensuring an eternal stream of bugs caused by programmers
    switching between it and other languages.}

\cite{Dijkstra1982} presents several justifications.  The set of natural
numbers, $\mathbb{N}$\footnotemark, has a minimum element (regardless of whether
it is assumed to include the number zero or not).  Left-open notations need to
include an element outside of the set to describe it.  Similarly, right-closed
notations need to include an element outside of the set to describe an empty
interval when the left endpoint is the minimum natural element (since, for
example, $[0, 0]$ is the unit set $\{0\}$).  This makes the half-open,
left-closed notation the preferred one for specifying ranges of values.  Beyond
these characteristics, it has several other notational advantages (some of which
are also true for the other notations in certain cases):

\footnotetext{
    Or, alternatively --- and anticipating the conclusion, for programmers: an
    \texttt{unsigned} integer type.}

\begin{itemize}
    \item the length of the interval $[x,y)$ is exactly $y - x$
        \begin{itemize}
            \item $[x,x)$ is an empty interval
            \item an interval of length $N$ can be represented as $[0,N)$
            \item
                an interval of length $N$ starting at $x$ can be represented
                as $[x,x+N)$
            \item $x \bmod N \in [0,N)$
        \end{itemize}
    \item two intervals $[x,y)$ and $[z,w)$ are adjacent if $y = z$ or $w = x$
        \begin{itemize}
            \item the concatenation of intervals $[x,y)$ and $[y,z)$ is $[x,z)$
            \item
                an interval $[0,N)$ split at the $n$-th element results in the
                intervals $[0,n)$ and $[n,N)$
            \item
                an interval $[x,y)$ split at the $n$-th element results in the
                intervals $[x,x+n)$ and $[x+n,y)$
            \item
                an interval $[x,y)$ which can be divided evenly into
                sub-intervals of size $n$ is the concatenation of the intervals
                $[x, x + n), [x + n, x + 2n), [x + 2n, x + 3n), \ldots$
        \end{itemize}
\end{itemize}

When applied to ranges of values in a computer program (e.g. sequences of memory
addresses), these properties eliminate most classes of the pernicious
\textit{off-by-one} errors.  For example, the trivial splitting property makes
partitioning a range (in the middle, say, for a binary search) much easier and
less prone to errors:

\begin{lstlisting}[style=c]
// Recursively partition the range [b,e) in half.
void f(int *b, int *e) {
    if(b == e)
        return;
    int *m = b + (e - b) / 2;
    use(m);
    f(b, m), f(m + 1, e);
}
\end{lstlisting}

Its other beneficial property results in a notation that may seem at first
unusual.  Consider the depiction of the array in figure \ref{fig:algo:range0},
and suppose we wish to represent the shaded interval, which consists of the
elements at position 1 and 2.  Both $[1,2]$ and $[1,3)$ seem like equally valid
options, i.e. it is not clear whether the end of the range is (or should be)
inclusive or exclusive.  Enumerating the positions \emph{between} the elements,
as in figure \ref{fig:algo:range1}, results in an unambiguous representation:
the interval $[1,3)$ denotes the range between indices 1 and 3.

\begin{figure}[ht]
    \centering
    \begin{subfigure}[h]{0.45\textwidth}
    \centering
    \begin{tikzpicture}[
        >={Stealth[round]},
        minimum width = 2em,
        minimum height = 2em,
    ]
        \matrix {
            \node (n0) [rectangle, draw] {}; &
            \node (n1) [rectangle, draw, fill = black!20] {}; &
            \node (n2) [rectangle, draw, fill = black!20] {}; &
            \node (n3) [rectangle, draw] {}; \\
        };
        \node at ($ (n0.south) - (0, 0.25) $) {0};
        \node at ($ (n1.south) - (0, 0.25) $) {1};
        \node at ($ (n2.south) - (0, 0.25) $) {2};
        \node at ($ (n3.south) - (0, 0.25) $) {3};
    \end{tikzpicture}
    \caption{Elements indicated by position}
    \label{fig:algo:range0}
\end{subfigure}
\quad
\begin{subfigure}[h]{0.45\textwidth}
    \centering
    \begin{tikzpicture}[
        >={Stealth[round]},
        minimum width = 2em,
        minimum height = 2em,
    ]
        \matrix {
            \node (n0) [rectangle, draw] {}; &
            \node (n1) [rectangle, draw, fill = black!20] {}; &
            \node (n2) [rectangle, draw, fill = black!20] {}; &
            \node (n3) [rectangle, draw] {}; \\
        };
        \node at ($ (n0.south west) - (0, 0.25) $) {0};
        \node at ($ (n1.south west) - (0, 0.25) $) {1};
        \node at ($ (n2.south west) - (0, 0.25) $) {2};
        \node at ($ (n3.south west) - (0, 0.25) $) {3};
        \node at ($ (n3.south east) - (0, 0.25) $) {4};
    \end{tikzpicture}
    \caption{Indices between elements}
    \label{fig:algo:range1}
\end{subfigure}

\end{figure}

Consider also how the end position in this type of half-open range mirrors the
form of the traditional \texttt{for} loop in a language such as C for not only
numbers and addresses, but any type in general:

\begin{lstlisting}[style=c]
for(int i = b; i < e; ++i)
for(int *p = b; p < e; ++p)
for(T x = b; x < e; ++x)
\end{lstlisting}

This is the basis of the rule in the C standard that allows a program to form
(but not dereference) pointers to \textit{one past the last element} of an
array, and it illustrates the rule more clearly.  More importantly for languages
such as C++, it is also equivalent to and enables the more general form:

\begin{lstlisting}[style=c]
for(int i = 0; i != N; ++i)
for(int *p = b; p != e; ++p)
for(T x = b; x != e; ++x)
\end{lstlisting}

This transformation --- which could not happen if $[b,e)$ were a closed interval
--- of \texttt{<} into \texttt{!=} may seem inconsequential, but is subtly
profound.  This is because it no longer limits this construct to contiguous
ranges\footnotemark: it can be applied to any range which supports a
\textit{sentinel} value as the endpoint, such as a tree or linked list.  This is
the foundation of iterators and ranges in C++, and the standard library
algorithms built on top of them.

\footnotetext{
    V. the \texttt{concept}s in the \texttt{std::ranges} namespace of the C++
    standard library, such as \texttt{forward\_range},
    \texttt{random\_access\_range}, and \texttt{contiguous\_range}.}

Even in less expressive languages, half-open ranges eliminate ambiguities.  An
insertion at index \texttt{i} positions the value between the elements indicated
by the index.  Inserting at position \texttt{0} means prepending, inserting at
position \texttt{N} means appending.  Denoting sub-ranges, e.g. for deletion, is
also trivial as shown above.

\subsection{Utilities}

Throughout this chapter, a number of utility functions will be used in the
implementation of algorithms.  Some will be described as part of each section,
but a few basic ones are presented here.

\subsubsection{\texttt{less\_than}}

A function which creates a function object (implemented using a lambda
expression) used in algorithms to order elements based on the less-than operator
(\texttt{<}).  One subtle aspect of the implementation is a copy of the original
value is stored, making it suitable as a predicate when the underlying range may
be changed after the object is created (e.g. in a sorting algorithm).

\lstinputlisting[style=c++,firstline=6,lastline=6]{algo/fundamentals/utils.hpp}
\vspace{-\baselineskip}

\subsubsection{\texttt{check\_range}}

Verifies that an iterator/sentinel pair represents a valid range.  It is meant
as a debugging aid and only used in assertions.  The specific test depends on
the types of the inputs:

\begin{itemize}
    \item
        Contiguous and random-access iterators are tested using
        \texttt{std::distance}, a constant-time operation for those types.
    \item
        A forward iterator needs to be advanced until the sentinel is reached,
        also done using \texttt{std::distance}.  This is an invalid operation
        for invalid ranges, but so will be their use in the algorithms where
        \texttt{check\_range} is used.  For this reason, the actual value of the
        distance does not matter.
    \item
        Input iterators cannot be checked, since they only yield their value
        once.
\end{itemize}

These cases can be nicely discriminated using constraints and concepts from the
standard library.

\lstinputlisting[style=c++,linerange=8-18]{algo/fundamentals/utils.hpp}
\vspace{-\baselineskip}

\subsubsection{\texttt{rand\_it}}

Returns an iterator in the range $[\texttt{b}, \texttt{e})$ chosen randomly
(using a global \texttt{std::random\_device} object for simplicity).

\lstinputlisting[style=c++,linerange=25-31]{algo/fundamentals/utils.hpp}
\vspace{-\baselineskip}

\subsubsection{\texttt{contains}}

Uses \texttt{check\_range} to determine whether an iterator points to a valid
element of a range.

\lstinputlisting[style=c++,linerange=20-23]{algo/fundamentals/utils.hpp}
\vspace{-\baselineskip}

\subsubsection{\texttt{is\_sorted}}

A C version of the same function in the C++ standard library.

\lstinputlisting[style=c,firstline=3,lastline=10]{algo/fundamentals/utils.h}
\vspace{-\baselineskip}

\subsubsection{\texttt{is\_min\_element}}

Verifies that a given value is the minimum element of a range of values --- i.e.
that no other element is less than it.  It is conceptually equivalent to
\texttt{x <= *std::min\_element(b, e)}, but uses the less-than operator and
stops at the first element for which the assertion does not hold.

\lstinputlisting[style=c++,linerange=33-36]{algo/fundamentals/utils.hpp}
\vspace{-\baselineskip}

\subsubsection{\texttt{min\_element}}

A reimplementation of \texttt{std::min\_element} for illustrative purposes.

\lstinputlisting[style=c++,linerange=38-52]{algo/fundamentals/utils.hpp}
\vspace{-\baselineskip}

\subsubsection{\texttt{sort\_element}}

Positions the element \texttt{i} in the correct position of the range if it were
sorted.  Similar to \texttt{std::nth\_element}, but based on the element's value
instead of its current position.  As a side-effect, the range is partitioned
based on \texttt{less\_than(*i)}.

\lstinputlisting[style=c++,linerange=54-60]{algo/fundamentals/utils.hpp}
\vspace{-\baselineskip}

\subsubsection{\texttt{lsearch}}

Used in contrast to \texttt{bsearch} in examples in C.

\lstinputlisting
    [style=c,firstline=12,lastline=17]
    {algo/fundamentals/utils.h}
\vspace{-\baselineskip}
