\section{Semaphores}

\label{subsec:conc:sem}

A semaphore is a concurrency primitive that limits access to a section of code.
It can be conceptualized as a simple unsigned integer and associated operations
to decrement and increment its value, traditionally called \textit{P} and
\textit{V}\footnotemark.

\footnotetext{
    The initials used by Edsger W. Dijkstra in the papers that introduced these
    concepts, thought to be Dutch words used in railway systems that roughly
    translate to ``passing'' (the noun) and ``release''.}

\begin{description}
    \item[P]
        decrements the value of the semaphore.  This operation can be performed
        as long as the value is greater than zero.  A thread which attempts to
        decrement a semaphore with value less than or equal to zero is
        suspended.  This operation is also often called \textit{acquiring} or
        \textit{locking} the semaphore.
    \item[V]
        increments the value of the semaphore.  If there are any suspended
        threads, one of them is awakened.  This operation is also often called
        \textit{releasing} or \textit{unlocking} the semaphore.
\end{description}

From this description, we derive that a semaphore is implemented in terms of a
counter, a list of suspended threads, and a lower-level primitive to synchronize
access to them (figure \ref{fig:conc:sem}.  A thread performing a decrement
first locks the semaphore, then checks the counter.  If it is zero, the thread
adds itself to the wait list and is suspended; otherwise, it continues
execution.  A thread performing an increment locks the semaphore, increments the
value, and pops one thread from the wait list (if it is not empty) and resumes
it.  In all cases, the thread releases the lock at the end of the operation or
before being suspended.

\begin{figure}[ht]
    \centering
    \begin{tikzpicture}[>={Stealth[round]}]
        \node (sem) [object = 4] {
            sem
            \nodepart{two}lock
            \nodepart{three}count
            \nodepart{four}wait
        };
        \node (l0) [lnode, right = of sem.four east] {};
        \node (l1) [lnode, right = of l0] {};
        \node (l2) [lnode, right = of l1] {};
        \draw[->] (sem.four east) -- (l0);
        \draw[->] (l0) -- (l1);
        \draw[->] (l1) -- (l2);
    \end{tikzpicture}
    \caption{Semaphore}
    \label{fig:conc:sem}
\end{figure}

The region of code between the semaphore decrement/increment pair of calls must
usually only be executed by a limited number of threads.  The initial value of
the semaphore and the semantics of the increment/decrement operations can be
used to satisfy this constraint.

\subsection{Mutex}

\label{subsec:conc:mutex}

Semaphores whose initial value is \emph{one} are the most common.  The region of
code such semaphore protects is called a \textit{critical section}, since the
value of the semaphore guarantees that only a single thread can execute it at
any point in time.  The code in the critical section is therefore effectively
\emph{atomic}: the observable behavior is that all of its executions are
serialized.  A semaphore of this type is referred to as a \textit{binary
semaphores} or \textit{mutex} (short for \textit{mutual exclusion}).

An atomic boolean flag can be used to implement the simplest possible (binary
spin-)lock (listing \ref{lst:conc:atomic_lock}).

\begin{lstlisting}[
    style=c++,
    label=lst:conc:atomic_lock,
    caption={Simple lock using atomic operations},
]
class mutex {
    using enum std::memory_order;
    std::atomic_flag f;
public:
    void lock(void) { while(this->f.test_and_set(acquire)); }
    void unlock(void) { this->f.clear(release); }
};
\end{lstlisting}

\subsection{R/W semaphore}

Both the \textit{P}/\textit{V} operations on a semaphore treat the calling
thread indiscriminately.  Often, however, not all scenarios require strict
mutual exclusion.  One of the most common is where an unlimited number of
\emph{readers} is allowed, but only one \emph{writer} should be allowed to enter
the critical section.  A \textit{read/write semaphore} is a concurrency
primitive that satisfies these constraints.  Reader threads can acquire the
semaphore without contention as long as there are no writers.  When a thread
wishes to acquire the semaphore for writing, it blocks new readers from
acquiring it and waits until all current readers are done.  From then on, it has
exclusive access to the semaphore.

This prioritization of writers means readers can be denied access for long
periods of time (i.e. \textit{reader starvation}).  For this reason, R/W
semaphores are best suited when reads are frequent and writes are fast and rare,
but can result in significantly better performance in those cases.

\subsection{\texttt{futex(2)}}

A \textit{futex} (short for \textit{fast user-space mutex}) is a primitive in
the Linux kernel used to build higher-level concurrency primitives, introduced
in 2003 in the 2.6 stable series by Hubertus Franke, Matthew Kirkwood, Ingo
Molnár, and Rusty Russell.  The ``fast'' portion of the name refers to its
user-space component, which can be as simple as a single atomic operation on an
integer.  The \texttt{futex(2)} system call handles the slow case of suspending
threads when there the lock is contended.  Futexes are fast because the cost of
a system call can be completely avoided when there is no
contention.\footnotemark

\footnotetext{
    V. section \secref{sec:concurrency:rcu} for another example of optimizing
    for the uncontended case.}

The wait queue of suspended threads exists on the kernel side and is associated
with the memory address supplied by user space.  The \texttt{futex(2)} system
call is a multiplexer (in the same manner as \texttt{ioctl(2)}) whose two main
operations are \texttt{FUTEX\_WAIT} and \texttt{FUTEX\_WAKE}.  \texttt{WAIT}
adds the current thread to the wait queue (after ensuring --- atomically ---
that the value being waited on has not changed), while \texttt{WAKE} resumes
threads currently on the queue.  Other operations are available as more
specialized versions of \texttt{WAKE}:

\begin{itemize}
    \item \texttt{FUTEX\_CMP\_REQUEUE}:
        wakes a specified number of threads and transfers any remaining ones to
        a second queue (avoids a \textit{thundering herd} on wake)
    \item \texttt{FUTEX\_WAKE\_OP}:
        provides a more flexible waking comparison operation based on a number
        of available operations on a second value
\end{itemize}

Higher-level concurrency primitives are built using these two operations.  One
of the simplest is an event that can be signaled and waited on (extracted,
slightly reformatted, from \cite{Drepper2011}, v. for further examples and
discussion):

\begin{lstlisting}[style=c++]
class event {
public:
    void signal(void) { futex_wake(&++this->val, INT_MAX); }
    void wait(void) { futex_wait(&this->val, this->val); }
private:
    int val = 0;
};
\end{lstlisting}

\subsection{Deadlock}

Whenever a thread attempts to acquire a lock, it may put itself in a situation
called a \textit{deadlock}.  This happens when the locking operation is blocking
and results in a situation where no sequence of operations will ever unblock the
thread and allow for forward progress.  Two common cases of deadlock are when a
thread attempts to acquire a lock it currently already holds, or when it needs
to acquire more than one lock.

A \textit{recursive lock} can be re-acquired even if a thread currently holds
it.  There is often a performance cost to this implementation, so regular locks
are generally not recursive.  Whether a thread which attempts to lock a
non-recursive lock while holding it immediately deadlocks or receives an error
is usually defined by the implementation, but the former case is more common.

Two general strategies exist for the case where multiple locks must be acquired.
The simplest solution is to establish a hierarchy of locks and guarantee that
they are always acquired in the same order by all threads.  This prevents
deadlocks since they are caused by two threads acquiring two different locks
then attempting to acquire the one just acquired by the opposite thread.
Another solution when a hierarchy cannot be easily established is to use the
``speculative locking'' operation commonly provided by implementations, where a
thread can attempt to acquire a lock without blocking if it is already taken.
Threads can use this operation when acquiring all locks after the first,
releasing all of them if it fails and retrying the entire operation.

\section{pthreads}

\subsection{Condition variable}

\cite{Kerrisk2010}, 30.2
