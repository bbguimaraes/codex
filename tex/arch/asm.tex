\section{Machine language}
\label{sec:arch:asm}

\subsection{Stack}

There are no "functions" per se in machine language\footnotemark.  Control flow
happens with \textit{jump} instructions, including the equivalent of a function
call.  The beginning of the call (equivalent to the initial expression followed
by parentheses, \texttt{f(…)}) can be a simple, unconditional jump.  The
destination address is either a fixed value, as determined by the linker either
at link- or runtime for static or dynamic linking, or a dynamic value (i.e. from
a register) in the case of an indirect call.  The end of the call (equivalent to
the \texttt{return} expression), conversely, is variable: it has to return to
the original place, right after the previous jump instruction.

\footnotetext{
    Regardless, this section will use terms such as "call" and "return" for
    simplicity of description.}

This is done via a region of memory called the \textit{stack}.  It contains
different types of data about the runtime execution of the program, as will be
described in this section, but the one which is immediately interesting is the
\textit{return address}.  Using the stack, the protocol for jumping to a new
code location and then back is:

\begin{itemize}
    \item
        Before the switch to the new location in code, the return address is
        pushed onto the stack.  This is the address of the next instructions
        after the imminent jump.
    \item
        When the target code is done and wishes to return to the previous
        location, it pops the return value from the stack and jumps to it.
\end{itemize}

This process can be arbitrarily nested: the stack maintains the sequence of
return addresses required to return to the previous location at each level,
limited only by the total size of the stack\footnotemark.

\footnotetext{Limited roughly by the total size of memory, or a multiple of it.}

The region of memory where the stack is located is a closed range
denoted by two registers\footnotemark:

\footnotetext{
    Implementation details in this section all use the x86 architecture as an
    example, but these concepts are universal and most other architectures have
    very similar concepts.}

\begin{description}
    \item[\texttt{rbp}]
        is the \textit{base pointer}, the starting location of the stack.
    \item[\texttt{rsp}]
        is the \textit{stack pointer}, the location of the last element.
\end{description}

\begin{description}
    \item[\texttt{push}]
        increments the pointer and writes a value to its location, equivalent
        to (where \texttt{x} is a 32-bit register/value):
        \begin{lstlisting}[style=x86]
add rsp, 4
mov [rsp], x
        \end{lstlisting}
    \item[\texttt{pop}]
        reads the value pointed to by the pointer and decrements the pointer,
        equivalent to (where \texttt{x} is a 32-bit register):
        \begin{lstlisting}[style=x86]
mov x, [rsp]
sub rsp, 4
        \end{lstlisting}
\end{description}
