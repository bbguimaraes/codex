\section{Atomic operations}

\label{sec:conc:atomic}

Section \secref{sec:conc:consistency} described several memory consistency
models adopted by different architectures.  It also briefly described how weaker
models provide specialized instructions to selectively and temporarily impose
stronger guarantees.  This section will present these operations, collectively
called \textit{atomic operations}, in more detail.

\subsection{Basic properties}

\label{subsec:conc:atomic_prop}

The term ``atomic'' is commonly (and often ambiguously) used to refer to several
different aspects of the underlying machine concepts and operations, each
explained in more detail in the following sections.

\subsubsection{\texttt{volatile}}

\texttt{volatile} type qualifiers act as compiler barriers, inhibiting most
optimizations.  ``Volatile'' is the name of a type qualifier (similar to
\texttt{const}) used in C/++ to indicate that a variable may be modified by
processes external to the thread of execution presented to the compiler, such as
external hardware or other operating system threads\footnotemark.  They prevent
the compiler from reordering loads and stores and from performing optimizations
such as read/write coalescing.

\footnotetext{
    \texttt{volatile} can have different meanings in other languages and even
    compilers.  In Java, it means sequential consistency (similar to operations
    on \texttt{std::atomic} types in C++ without explicit memory ordering).
    Microsoft's compiler adds non-standard acquire/release semantics to
    loads/stores (unconditionally in older versions, now, by default).}

In some architectures, this is sufficient to implement relaxed atomic operations
(see below) for common types\footnotemark.  For this reason, it is found often
in platform-specific code which predates the C/++ 11 memory model.  When
available, the atomic types introduced by that model are preferred, as they
offer more control over which optimizations the compiler is allowed to perform.
Code which uses \texttt{volatile} also often has to resort to more expensive,
general fence instructions instead of the specific acquire/release operations
available in some architectures.

\footnotetext{
    E.g. the \texttt{READ\_ONCE} and \texttt{WRITE\_ONCE} macros in the Linux
    kernel, which have relaxed ordering but guaranteed atomicity, are
    essentially \texttt{volatile} loads/stores.  See also
    \texttt{process/volatile-considered-harmful.rst} in the kernel
    documentation.}

\subsubsection{Atomic loads/stores}

These provide safe reads and writes by multiple threads.  These are special
(depending on the architecture) load/store operations which are guaranteed to be
performed indivisibly.  An external observer will either not see them, or see
their result in its entirety: no partial result is allowed.  In some
architectures, such as x86, unadorned loads and stores up to a given size and
properly aligned behave in this way.  In other cases, specialized instructions
or external synchronization may be required to avoid torn loads/stores.

In languages such as C/++ which provide low-level control of the memory ordering
of atomic operations, regular atomic loads/stores that \emph{do not} impose any
ordering (i.e. are used just for their atomicity property) are called
\textit{relaxed} atomic operations.  While inter-thread synchronization requires
ordering constraints, relaxed atomic operations are useful in cases where no
synchronization is needed or it is already provided via other means.

\subsubsection{Acquire/release semantics}

The main mechanism for memory ordering with atomic instructions in programming
languages are operations with \textit{acquire/release semantics}, which are
counterparts that constrain the reordering of operations and synchronize memory
accesses across threads.

\begin{description}
    \item[Release semantics]
        prevent reordering of operations preceding the atomic operation with
        itself.  Other processors are guaranteed to see the result of the
        preceding operations before succeeding operations.
    \item[Acquire semantics]
        prevent reordering of operations succeeding the atomic operation with
        itself.  Other processors are guaranteed to see the result of the
        succeeding operations before preceding operations.
\end{description}

Note that each of these guarantees is unidirectional: operations can be moved
down or up \emph{inside} a critical section.  Normally, this is
counterproductive since it increases contention but may be beneficial when
combined with other optimizations.

Acquire and release semantics are an important building block of concurrent
algorithms because of their transitivity.  This means they can be a hidden
implementation detail of certain operations that in turn can be thought of as
themselves having acquire/release semantics.  Algorithms can then be built using
those operations and reasoned about at a higher level.

The most common example is starting and joining threads.  A function such as
\texttt{pthread\-\_create} has release semantics, meaning all memory accesses
are guaranteed to be seen by the thread that is being started.  Equivalently,
\texttt{pthread\_join} has acquire semantics, meaning all memory accesses of the
thread that just finished are guaranteed to be seen by the waiting thread.  This
means algorithms where producers and consumers are separated by this type of
fork-join boundary can often operate with no locks or even atomic operations at
all: the implicit acquire/release semantics of starting and joining the worker
threads is enough to guarantee safe access to shared data.

A lock or mutex (v. section \secrefpar{subsec:conc:mutex}) is another common
example: releasing a lock is done with release semantics so that operations in
the critical section are guaranteed to be seen before the release, acquiring a
lock is done with acquire semantics so that the acquisition is guaranteed to be
seen before the operations in the critical section.

\subsubsection{Fence/barrier instructions}

These are machine instructions which establish memory ordering guarantees
between different threads operating on the same data.  They are the low-level
mechanism that enables acquire/release semantics.

Some programming languages (such as C/++) provide fence operations in addition
to the regular atomic types.  Acquire and release fences are similar to atomic
operations with acquire and release semantics and provide similar guarantees.
The one difference between them is an atomic operation is tied to the variable
it is invoked on, so it prevents reorderings in relation to \emph{itself}; a
fence has no attached variable, so it prevents reorderings in relation to
\emph{all operations that precede or succeed it}, for acquire and release
fences, respectively.

\subsubsection{Read-modify-write}

These operations are usually what is meant by ``atomic operations'' stricto
sensu.  These combine atomic loads/stores and fence/barrier instructions to
modify shared data safely in multiple threads.

\subsection{Optimization}

Single-threaded programs give the compiler and the hardware many opportunities
to transform the operations as listed in source code to better exploit machine
capabilities and achieve better performance.  It is worth analyzing these sorts
of optimizations for a better understanding of why they have to be suppressed
for the correct functioning of multi-threaded.  They will be described from the
point of view of the compiler, but for the most part are equivalent --- at the
source-code level --- to transformations done by other components at the
hardware level.  Consult section \secref{sec:conc:consistency} for a detailed
description of the fundamentals that make these transformations legal under
traditional memory models.

A compiler which encounters a series of redundant loads from a variable can
decide to coalesce them to a single load:

\begin{center}
    \begin{tabular}{ccc}
        \begin{lstlisting}[style=c]
x = g;
x = g;
x = g;
        \end{lstlisting}
        & $\to$ &
        \begin{lstlisting}[style=c]
x = g;
        \end{lstlisting} \\[1.25em]
    \end{tabular}
\end{center}

This is because, in the absence of data races (which are expressively forbidden
by language memory models), there are no legal ways in which the value of
\texttt{g} could change between each load.  Similarly, multiple redundant stores
to the same variable can be coalesced to a single store of the final value.
These optimizations can affect control structures as well:

\begin{center}
    \begin{tabular}{ccc}
        \begin{lstlisting}[style=c]
while(g)
    use(g);
        \end{lstlisting}
        & $\to$ &
        \begin{lstlisting}[style=c]
if(tmp = g)
    for(;;)
        use(g);
        \end{lstlisting}
    \end{tabular}
\end{center}

All of these cases are legal transformations because single-threaded behavior is
maintained.  They are catastrophic in multi-threaded code, however, where shared
data accesses are used for inter-thread communication.  Reorderings of this type
must be prevented in some cases for an algorithm to function correctly.
Conversely, it is desirable to do this \emph{only when necessary}.  These
optimizations happen for a good reason, and excessive suppression can result in
inefficient programs.  For example, the compiler might decide to make the
following transformation:

\begin{center}
    \begin{tabular}{ccc}
        \begin{lstlisting}[style=c]
while(tmp = g)
    use(tmp);
        \end{lstlisting}
        & $\to$ &
        \begin{lstlisting}[style=c]
while(g)
    use(g);
        \end{lstlisting}
    \end{tabular}
\end{center}

This may happen due to register exhaustion; forbidding such transformation (e.g.
by making \texttt{g} \texttt{volatile}) will result in \texttt{tmp} being
spilled on the stack instead of the double load from \texttt{g}.  Also note
that, although there was no code surrounding the relevant operations in these
examples, optimizations of operations involving other variables may also be
affected.

Counter-intuitively, stores can also be added by the compiler, such as the
following example, where a branch is removed with the addition of a second
store:

\begin{center}
    \begin{tabular}{ccc}
        \begin{lstlisting}[style=c]
if(i)
    g = i;
else
    g = 42;
        \end{lstlisting}
        & $\to$ &
        \begin{lstlisting}[style=c]
g = 42;
if(i)
    g = i;
        \end{lstlisting}
    \end{tabular}
\end{center}

\subsection{Wait-free queue}

One of the fundamental constructs in multi-threaded code is the FIFO queue.
Many problems are structure in some form of producer-consumer relation, and a
queue is often a good mechanism for communication between them.  This section
will demonstrate the use of atomic operations in the construction of concurrent
queues of varying complexity.

\begin{figure}[ht]
    \centering
    \begin{subfigure}[t]{0.5\textwidth}
        \lstinputlisting[
            style=c,
            firstline=5,
            label=lst:conc:queue_st,
            caption={Single-threaded queue},
        ]{concurrency/queue_st.c}
    \end{subfigure}
    \begin{subfigure}[t]{0.45\textwidth}
        \lstinputlisting[
            style=c,
            firstline=6,
            label=lst:conc:queue_mtx,
            caption={Queue with mutex},
        ]{concurrency/queue_mtx.c}
    \end{subfigure}
\end{figure}

\subsubsection{Single-threaded}

For illustrative purposes, we start with an extremely constrained implementation
(listing \ref{lst:conc:queue_st}), which
\begin{enumerate*}[1)]
    \item is not used concurrently,
    \item only stores \texttt{int}s,
    \item has a fixed size,
    \item is either read- or write-only, and
    \item can be read from / written to only once
\end{enumerate*}

All these restrictions are imposed so that this first implementation can be
almost non-existent and serve as a base for the incremental addition of
concurrent facilities.  Still, there are many scenarios in which a data buffer
is read and/or written iteratively but received and/or sent in its entirety
where this data structure is already useful.  As mentioned in the description of
acquire/release semantics in section \secref{subsec:conc:atomic_prop}, the
implicit guarantees of starting and joining threads is often enough to properly
synchronize memory accesses in these cases.

Other than the buffer where the data are stored, the only other member is an
index into this buffer, the read/write pointer.  In both modes, checking if the
queue is full ---  i.e. whether values can still be pushed/popped --- is simple,
since this is a single-threaded queue.  In a read-only queue the only other
operation is \texttt{pop}, which loads the current value and advances the
pointer.  Conversely, in a write-only queue the only other operation is
\texttt{push}, which stores the value and advances the pointer.

\subsubsection{Read-/write-only}

The first concurrent application we will examine is multiple producers or
multiple consumers.  A queue is still read-/write-only, but multiple threads are
now allowed to populate the buffer in parallel.  This presents a problem as the
expression \texttt{q->i++}, even though it is a single statement in C, is
composed of multiple operations.  Of relevance here is how the queue index is
incremented in each function:

\begin{multicols}{2}
    \begin{lstlisting}[style=x86,showlines=true]
push:
    mov eax, [rdi]
    lea edx, 1[rax]
    mov [rdi], edx
    mov 4[rdi+rax*4], esi
    ret

pop:
    mov eax, [rdi]
    lea edx, 1[rax]
    mov eax, 4[rdi+rax*4]
    mov [rdi], edx
    mov [rsi], eax
    ret

    \end{lstlisting}
    \columnbreak
    \begin{lstlisting}[style=arm]
push:
        ldr r3, [r0]
        add r2, r3, #1
        add r3, r0, r3, lsl #2
        str r2, [r0]
        str r1, [r3, #4]
        bx  lr
pop:
        ldr r3, [r0]
        add r2, r0, r3, lsl #2
        ldr r2, [r2, #4]
        add r3, r3, #1
        str r3, [r0]
        str r2, [r1]
        bx  lr
    \end{lstlisting}
\end{multicols}
\vspace{-\baselineskip}

Incrementing the value of the index in memory involves separate load, increment,
and store instructions, seen at the beginning of the x86 version of
\texttt{push} in the listing above\footnotemark.  Since multiple threads may
execute this sequence of instructions and could be interleaved in any order,
there is the potential of data races and incorrect results (consider the case
where threads are suspended before they can execute the third instruction and
later resumed).  For this queue to operate correctly, the increment of the index
must be a serialized, atomic operation.

\footnotetext{
    Note also that GCC in this optimized build (\texttt{gcc -O2}) chose to first
    store \texttt{q->i} \emph{then} load/store the value \texttt{q->v} in both
    functions in both architectures.  We will analyze this important fact in
    later implementations.}

Naturally, one implementation possibility is to wrap operations with a mutex (v.
section \secrefpar{subsec:conc:mutex}), as shown in listing
\ref{lst:conc:queue_mtx}.  This guarantees serialization and may be appropriate
under low contention (v. \cite{Preshing2011}).  In high-throughput queues,
however, the system calls required to suspend and wake threads can become a
limiting factor.

Before analyzing an alternative implementation, we first look at a problem that
follows from the introduction of multiple threads.  In a single-threaded
program, mere program order guarantees that consuming the data buffer is safe
after the last \texttt{push} operation finishes or before the first \texttt{pop}
starts.  No such guarantee exists in a multi-threaded scenario, so memory
ordering instructions are necessary.  In these examples, unless otherwise noted,
this ordering is assumed to be established by the operations which start the
threads and wait for them to finish --- i.e. that they have, respectively,
acquire and release semantics.

Since there are only ever producers \emph{or} consumers operating on the queue,
it is enough to guarantee that the increments are serialized and the
loads/stores are not reordered.  Listing \ref{lst:conc:queue_atomic} shows an
implementation using a relaxed atomic increment operation, which has all of
these properties.

\begin{figure}[ht]
    \lstinputlisting[
        style=c,
        firstline=5,
        label=lst:conc:queue_atomic,
        caption={Queue with atomic operations},
    ]{concurrency/queue_atomic.c}
\end{figure}

\subsubsection{\texttt{is\_full}}

Lock-free algorithms are rarely completely general and thus must be built to fit
particular usage and performance requirements.  Different circumstances often
require different algorithms, sometimes radically so.  The current
implementation has made \texttt{is\_full} purely informational: it cannot be
used to determine if the other operations are safe since the index may be moved
between the check and the subsequent increment.  It is assumed that there are
other ways of guaranteeing that operations do not read/write outside the buffer
(e.g. each thread is given an exact number of push/pop operations).

One alternative is to check the value returned by \texttt{atomic\_fetch\_add}
before performing the load/store.  The atomic increment has already ensured the
thread has a unique position in the buffer by the time the check is done, so no
race condition is introduced.  The queue index is always incremented, however,
so this is only safe if there are enough values between \texttt{N} and
\texttt{UINT\_MAX} (the maximum value for \texttt{q->i}) so that all threads can
increment it once and see that it is now out of bounds.  If not, the value will
eventually wrap back to zero\footnotemark and appear to be valid again.

\footnotetext{
    N.b.: \texttt{q->i} is an \texttt{unsigned} value, so this is defined
    behavior.  See also the discussion below of the ABA problem introduced by
    reusing index values.}

Listing \ref{lst:conc:queue_cas} shows a more robust implementation which
eliminates these restrictions at the cost of more expensive atomic operations.
It uses an \textit{atomic compare-and-swap} (CAS) loop to ensure \texttt{q->i}
is only assigned valid indices.  A compare-exchange operation has three
arguments (other than the memory ordering specification): the destination and
the \emph{expected} and \emph{desired} values.  It first compares the
destination and expected values.  If they are equal (i.e. the destination has
not been modified by other threads since it was last loaded), it stores the
desired value in the destination.  If they are not, the expected value is
updated with the current value of the destination.  All of these steps happen as
a single atomic operation.

\begin{figure}[ht]
    \lstinputlisting[
        style=c,
        firstline=11,
        lastline=28,
        label=lst:conc:queue_cas,
        caption={Queue with atomic compare-and-swap},
    ]{concurrency/queue_cas.c}
\end{figure}

The queue index is first loaded using a relaxed atomic operation.  If the value
is ever observed to be at the end of the buffer, the \texttt{for} loop
terminates and \texttt{NULL} is returned.  In each iteration, we increment the
value just loaded and attempt to store it, checking to see if it has not changed
while it was being tested.  This type of construction, where an atomic CAS is
used to build an effectively atomic operation out of one or more non-atomic
ones, is a very common pattern in lock-free algorithms.

\subsubsection{Ring buffer}

An alternative design is to let threads continuously operate on the queue even
when there is no space left.  Whenever the index reaches the end of the queue,
it is reset to the beginning (conceptually, at least).  This arrangement is
usually called a \textit{ring buffer}, since the wrapping of the index space can
be conceptualized as a circle (\textit{circular buffer} is also commonly used).

Push operations overwrite older values when there is no space left on the queue.
This may be desirable if only the last $N$ or less items are relevant, which is
common, for example, in real-time systems.  The pop equivalent is less common,
but may be useful if threads consume a finite, pre-calculated set of data.  When
the buffer is exhausted, the index is simply reset to the beginning and threads
start reusing old values, potentially in a different order.

Listings \ref{lst:conc:queue_ring_mod} and \ref{lst:conc:queue_ring_and} show
implementations of this pattern.  The only change is to the \texttt{advance}
function, since it is the one which defines the index space.  \texttt{is\_full}
has no meaning in ring buffers, so it is not included.  Listing
\ref{lst:conc:queue_ring_mod} is implemented in a general way, using the integer
modulus operation.  Since this is a relatively expensive operation, a common
choice is to limit the queue size to powers of 2.  The remainder can then be
found with a simple bitwise \texttt{and} operation, which is usually an order of
magnitude faster.  Listing \ref{lst:conc:queue_ring_and} shows this
implementation\footnotemark.

\footnotetext{
    V. section \ref{subsubsec:arch:sub} for an explanation of the expression
    used in the \texttt{static\_assert}.}

\begin{figure}[ht]
    \begin{subfigure}[t]{0.5\textwidth}
        \lstinputlisting[
            style=c,
            firstline=11,
            lastline=17,
            label=lst:conc:queue_ring_mod,
            caption={Ring buffer},
        ]{concurrency/queue_ring_mod.c}
    \end{subfigure}
    \begin{subfigure}[t]{0.45\textwidth}
        \lstinputlisting[
            style=c,
            firstline=12,
            lastline=19,
            label=lst:conc:queue_ring_and,
            caption={Ring buffer with bitwise \texttt{and}},
        ]{concurrency/queue_ring_and.c}
    \end{subfigure}
\end{figure}

\subsubsection{ABA}

This is our first encounter with the infamous \textit{ABA problem}, so called
after the case where a thread observes value \texttt{A} and is suspended, the
value is changed from \texttt{A} to \texttt{B} then back to \texttt{A}, and the
original thread is resumed.  Even though there was no change from the
perspective of the waking thread, the two values of \texttt{A} may not represent
the same state.  The canonical example is that of a pointer \texttt{A} to an
object which is deallocated and whose memory block is subsequently
reused\footnotemark.  A thread using the memory address contained in \texttt{A}
as an identifier will perceive no change, even though the object to which it
points is now different.

\footnotetext{
    It is common for memory allocators to use some kind of MRU policy since it
    increases the chance of maintaining blocks in cached memory.}

Our ring buffer suffers from a different type of ABA problem, stemming from the
fact that there is an interval between the calculation of the new index by
\texttt{advance} and the store in \texttt{push}.  All previous examples relied
on the fact that indices are not reused.  This is also why only the access to
the queue index is done using atomic operations: loads and stores to a
particular buffer position are always done by a single thread.  This is not true
in a ring buffer, but the serialization of the atomic increment operation is
used to ensure that each thread issues stores to a memory location it has
exclusive access to.

However, it is possible that a thread could be suspended right after calculating
its index and sleep while other threads go through a full rotation of the ring.
When the thread is awakened, it could attempt to write to the memory location it
originally calculated at the same time as another thread which just happened to
be assigned the same index, now a full rotation ahead.  Whether this is a
problem in reality is a function of the queue size and frequency of operations.
The larger the queue size the less likely a thread will be suspended for long
enough for the same index to be reused.  There are only a few machine
instructions involved, so the window for thread preemption is small:

\begin{multicols}{2}
    \begin{lstlisting}[style=x86]
push:
    mov       eax, 1
    lock xadd DWORD PTR [rdi], eax
    and       eax, 1023
    mov       4[rdi+rax*4], esi
    ret
    \end{lstlisting}
    \columnbreak
    \begin{lstlisting}[style=arm]
push:
.L5:
    ldrex r3, [r0]
    add   r2, r3, #1
    strex ip, r2, [r0]
    cmp   ip, #0
    bne   .L5
    ubfx  r3, r3, #0, #10
    add   r0, r0, r3, lsl #2
    str   r1, [r0, #4]
    bx    lr
    \end{lstlisting}
\end{multicols}
\vspace{-\baselineskip}

\subsubsection{Single-producer, single-consumer}

\begin{figure}[ht]
    \begin{subfigure}[t]{0.475\textwidth}
        \lstinputlisting[
            style=c,
            firstline=5,
            label=lst:conc:queue_spsc0,
            caption={(Incorrect) SP/SC queue},
        ]{concurrency/queue_spsc0.c}
    \end{subfigure}
    \begin{subfigure}[t]{0.5\textwidth}
        \lstinputlisting[
            style=c,
            firstline=6,
            label=lst:conc:queue_spsc1,
            caption={SP/SC queue},
        ]{concurrency/queue_spsc1.c}
    \end{subfigure}
\end{figure}

The queue implementations examined so far required no thread synchronization:
since only producers \emph{or} consumers were involved, it was enough to ensure
that the operations on shared memory were atomic (i.e. indivisible).  The
implicit serialization of the writes to the queue index was all that was needed
to arbitrate access to the buffer.

We now consider a single-producer, single-consumer queue (listing
\ref{lst:conc:queue_spsc0}).  There are now two indices: \texttt{r} and
\texttt{w}, the read and write pointers.  They are the equivalent of the
previous single index \texttt{i} for the consumer and the producer,
respectively, but allow them to share access to the buffer and operate at
different frequencies.  The write pointer is always at or ahead of the read
pointer.  We will initially return to a single-use queue, so in this example
both pointers advance only once to the end of the buffer.  We will later
reconsider ring buffers.

The implementation as shown is incorrect because relaxed atomic operations are
no longer sufficient: the queue now requires memory ordering operations to
synchronize the producer and the consumer.  To understand why, we must analyze
the sequence of operations performed by each in \texttt{push} and \texttt{pop}:

\begin{itemize}
    \item
        The read pointer is only ever modified by the (single) consumer in
        \texttt{pop}, so it is a regular \texttt{unsigned} variable.
    \item
        The write pointer is similarly only ever modified by the (single)
        producer in \texttt{push}, but this variable is also read by the
        consumer.  We signal this by the use of \texttt{volatile}\footnotemark.
    \footnotetext{
        Assuming loads/stores of \texttt{unsigned} values are atomic, relaxed
        atomic operations would be the portable mechanism.}
    \item
        To pop a value from the buffer, the consumer verifies that the producer
        has made data available, loads it, and advances the read pointer.
    \item
        To push a value to the buffer, the producer verifies that there is still
        space remaining (this is now a valid check as there is a single
        producer), stores it, and advances the write pointer.
    \item
        Loads and stores to the indices and the buffer must happen in program
        order, otherwise the consumer could see stale or uninitialized data.
\end{itemize}

Consider what can happen without memory ordering restrictions in a
weakly-ordered architecture.  The two stores in \texttt{push} could be effected
by the hardware in the opposite order\footnotemark:

\footnotetext{
    The \texttt{volatile} type qualifier of \texttt{q->w} is only applied at the
    language/compiler level.  It does not affect the types of loads/stores
    ultimately performed by the machine.}

\begin{center}
    \begin{tabular}{ccc}
        \begin{lstlisting}[style=c]
q->v[w] = i;
q->w = w + 1;
        \end{lstlisting}
        & $\to$ &
        \begin{lstlisting}[style=c]
q->w = w + 1;
q->v[w] = i;
        \end{lstlisting} \\[0.5em]
    \end{tabular}
\end{center}

If the reader side is executed between these two conceptual lines, it will read
the updated write position and load stale or uninitialized memory.  Similarly,
the two loads in \texttt{pop} could see values stored at different points in
time, each from a separate call to \texttt{push}:

\begin{center}
    \begin{tabular}{ccc}
        \begin{lstlisting}[style=c]
const unsigned w = q->w, r = q->r;
// ...
*i = q->v[q->r];
        \end{lstlisting}
        & $\to$ &
        \begin{lstlisting}[style=c]
*i = q->v[q->r];
const unsigned w = q->w, r = q->r;
// ...
        \end{lstlisting} \\[1.5em]
    \end{tabular}
\end{center}

Listing \ref{lst:conc:queue_spsc1} shows a revised implementation that
eliminates these problems using explicit memory ordering operations.  The
corresponding atomic load/store of \texttt{q->w} in \texttt{push}/\texttt{pop}
guarantee thread synchronization.  The fence operation with \emph{acquire}
semantics guarantees that all loads that succeed it see values stored after the
corresponding store was executed\footnotemark.  The store and all operations
preceding it are said to \textit{happen before} the fence and all operations
that succeed it.  This relationship is denoted with the right arrow ($\to$)
symbol: $l \to s$, load $l$ happens before store $s$.

\footnotetext{
    The acquire operation is not needed in the case where there is no data to be
    read, so a combination of a relaxed load + fence is used.  A single
    load/acquire operation would be equally valid, but with a potential
    efficiency loss.  The load of \texttt{q->w} at the beginning of
    \texttt{push} can be non-atomic since it is only ever written to by the
    (single) producer.}

This eliminates the first problem described above as the two instructions cannot
be reordered\footnotemark:

\footnotetext{
    In this and the following diagrams, the long arrow symbol
    ($\longrightarrow$) indicates the ``happens before'' relation described
    above between synchronized pairs of loads/stores in the code fragments
    immediately to its left and right.}

\begin{center}
    \begin{tabular}{ccc}
        \begin{lstlisting}[style=c,showlines=true]
q->v[w] = i;
q->w = w + 1;

        \end{lstlisting}
        & \parbox[c]{2em}{
            \vspace{0.5\baselineskip}
            $\longrightarrow$
        } &
        \begin{lstlisting}[style=c]

const unsigned w = /*...*/;
*i = q->v[r];
        \end{lstlisting} \\[1.25em]
    \end{tabular}
\end{center}

That is, any time the reader side loads a value into its \texttt{w} local
variable, all instructions following that load are guaranteed to see all stores
that preceded the corresponding store.  It will never see the store to
\texttt{q->v[w]} on the writer side without also seeing the
immediately-preceding store to \texttt{q->w}.  The load operation synchronizes
with the store, creating a barrier around which stores cannot be reordered.

It is important to note that this relationship exists only between two
corresponding load/store operations.  A \texttt{pop} operation that occurs in
between a series of \texttt{push}es could see any of the possible values of
\texttt{q->w}, depending on how the producer/consumer are interleaved.  The only
guarantee provided by memory ordering operations in this respect is that a
subsequent load will not see previous values of \texttt{q->w} (or any other
preceding store).

\begin{center}
    \begin{tabular}{ccc}
        \begin{lstlisting}[style=c,showlines=true]
q->v[w] = i;
q->w = w + 1;
q->v[w] = i;
q->w = w + 1;
q->v[w] = i;
q->w = w + 1;
q->v[w] = i;
q->w = w + 1;

        \end{lstlisting}
        & \parbox[c]{2em}{
            \vspace{2\baselineskip}
            $\longrightarrow$

            \vspace{2.25\baselineskip}
            $\longrightarrow$
        } &
        \begin{lstlisting}[style=c,showlines=true]



const unsigned w = /*...*/;
*i = q->v[r];


const unsigned w = /*...*/;
*i = q->v[r];
        \end{lstlisting} \\[1.5em]
    \end{tabular}
\end{center}

Similarly, this ``happens before'' relation is only a \emph{strong partial
order} (v. section \secrefpar{subsec:algo:ordering}): it establishes no relation
between two operations that happen before a third.  In other words, two
operations are concurrent if $\neg ((a \to b) \lor (b \to a))$.  It is, however,
transitive: $(a \to b) \land (b \to c) \, \implies \, a \to c$.  This can be
used to build a causal chain of arbitrary length between successive operations
in two or more threads, each of which is guaranteed to occur after all the
operations that happen before it.

\subsection{Code generation}

This section will explore the concepts discussed so far in this chapter by
analyzing the machine code generated for several architectures for the following
C code\footnotemark:

\footnotetext{
    Machine code generated using:
    \begin{itemize}
        \item x86: \texttt{gcc -std=c11 -O2} (11.2)
        \item ARMv7: \texttt{clang -march=armv7 -std=c11 -O2} (11.0.1)
        \item ARMv8: \texttt{gcc -march=armv8-a -std=c11 -O2} (11.2)
        \item PPC: \texttt{gcc -std=c11 -O2} (9.2.1)
    \end{itemize}}

\begin{lstlisting}[style=c]
#include <stdatomic.h>

atomic_int x;
int load_rlx(void) { return atomic_load_explicit(&x, memory_order_relaxed); }
int load_acq(void) { return atomic_load_explicit(&x, memory_order_acquire); }
int load(void) { return atomic_load(&x); }
void store_rlx(int i) { atomic_store_explicit(&x, i, memory_order_relaxed); }
void store_rel(int i) { atomic_store_explicit(&x, i, memory_order_release); }
void store(int i) { atomic_store(&x, i); }
void xchg_w(int i0, int i1) { atomic_compare_exchange_weak(&x, &i0, i1); }
void xchg_s(int i0, int i1) { atomic_compare_exchange_strong(&x, &i0, i1); }
\end{lstlisting}

\subsubsection{x86}

The x86 architecture provides the very strict TSO model.  The only reordering
allowed is that loads can proceed while stores are being processed.  In
particular, loads are not reordered with any other loads, stores are not
reordered with any other stores, and stores are not reordered with previous
loads.

This means there is no difference between regular loads and load/acquire
operations: they both translate to the \texttt{mov} instruction, and so all
\texttt{load} and \texttt{store} functions, except for the sequentially
consistent store, are simple \texttt{mov}s.  The store/release and
compare-exchange operations do need specialized instructions, \texttt{xchg} /
\texttt{lock cmpxchg}, which act as a full barrier (so there is no difference
between the weak and strong CAS).

\begin{multicols}{2}
    \begin{lstlisting}[style=x86,xleftmargin=0px,xrightmargin=0px]
load_rlx: # same for load_acq, load
    mov eax, DWORD PTR x[rip]
    ret
store_rlx: # same for store_rel
    mov DWORD PTR x[rip], edi
    ret
    \end{lstlisting}
    \columnbreak
    \begin{lstlisting}[style=x86,xleftmargin=0px,xrightmargin=0px]
store:
    xchg edi, DWORD PTR x[rip]
    ret
xchg_w: # same for xchg_s
    mov eax, edi
    lock cmpxchg DWORD PTR x[rip], esi
    ret
    \end{lstlisting}
\end{multicols}

\subsubsection{ARM}

When considering the ARM architecture, revision ARMv8-A and its successors have
to be analyzed separately.  That is because they introduced specialized
instructions dedicated to implementing release consistency to conform to the
C/++11 sequentially consistent atomic types:

\begin{multicols}{3}
    \begin{lstlisting}[style=arm,xleftmargin=0px,xrightmargin=0px]
load_rlx:
    movw r3, #:lower16:x
    movt r3, #:upper16:x
    ldr r0, [r3]
    bx lr
    \end{lstlisting}
    \columnbreak
    \begin{lstlisting}[style=arm,xleftmargin=0px,xrightmargin=0px]
load_acq:
    movw r3, #:lower16:x
    movt r3, #:upper16:x
    lda r0, [r3]
    bx lr
    \end{lstlisting}
    \columnbreak
    \begin{lstlisting}[style=arm,xleftmargin=0px,xrightmargin=0px]
load:
    movw r3, #:lower16:x
    movt r3, #:upper16:x
    lda r0, [r3]
    bx lr
    \end{lstlisting}
\end{multicols}

\begin{multicols}{3}
    \begin{lstlisting}[style=arm,xleftmargin=0px,xrightmargin=0px]
store_rlx:
    movw r3, #:lower16:x
    movt r3, #:upper16:x
    str r0, [r3]
    bx lr
    \end{lstlisting}
    \columnbreak
    \begin{lstlisting}[style=arm,xleftmargin=0px,xrightmargin=0px]
store_rel:
    movw r3, #:lower16:x
    movt r3, #:upper16:x
    stl r0, [r3]
    bx lr
    \end{lstlisting}
    \columnbreak
    \begin{lstlisting}[style=arm,xleftmargin=0px,xrightmargin=0px]
store:
    movw r3, #:lower16:x
    movt r3, #:upper16:x
    stl r0, [r3]
    bx lr
    \end{lstlisting}
\end{multicols}

Prior versions, such as ARMv7, require full memory barriers to implement that
model:

\begin{multicols}{3}
    \begin{lstlisting}[style=arm,xleftmargin=0px,xrightmargin=0px]
load_rlx:
    movw r3, #:lower16:x
    movt r3, #:upper16:x

    ldr r0, [r3]

    bx lr
    \end{lstlisting}
    \columnbreak
    \begin{lstlisting}[style=arm,xleftmargin=0px,xrightmargin=0px]
load_acq:
    movw r3, #:lower16:x
    movt r3, #:upper16:x

    ldr r0, [r3]
    dmb ish
    bx lr
    \end{lstlisting}
    \columnbreak
    \begin{lstlisting}[style=arm,xleftmargin=0px,xrightmargin=0px]
load:
    movw r3, #:lower16:x
    movt r3, #:upper16:x
    dmb ish
    ldr r0, [r3]
    dmb ish
    bx lr
    \end{lstlisting}
\end{multicols}

\begin{multicols}{3}
    \begin{lstlisting}[style=arm,xleftmargin=0px,xrightmargin=0px]
store_rlx:
    movw r3, #:lower16:x
    movt r3, #:upper16:x

    str r0, [r3]

    bx lr
    \end{lstlisting}
    \columnbreak
    \begin{lstlisting}[style=arm,xleftmargin=0px,xrightmargin=0px]
store_rel:
    movw r3, #:lower16:x
    movt r3, #:upper16:x
    dmb ish
    str r0, [r3]

    bx lr
    \end{lstlisting}
    \columnbreak
    \begin{lstlisting}[style=arm,xleftmargin=0px,xrightmargin=0px]
store:
    movw r3, #:lower16:x
    movt r3, #:upper16:x
    dmb ish
    str r0, [r3]
    dmb ish
    bx lr
    \end{lstlisting}
\end{multicols}

Also contrary to x86, strong compare-exchange is implemented as a loop in
machine code, meaning the weak version can be advantageous when it is already
surrounded by another loop:

\begin{multicols}{2}
    \begin{lstlisting}[style=arm]
xchg_w:
    movw r3, #:lower16:x
    movt r3, #:upper16:x

    ldaex r2, [r3]
    cmp r2, r0

    stlexeq ip, r1, [r3]

    bx lr
    \end{lstlisting}
    \columnbreak
    \begin{lstlisting}[style=arm]
xchg_s:
    movw r3, #:lower16:x
    movt r3, #:upper16:x
.L11:
    ldaex r2, [r3]
    cmp r2, r0
    bxne lr
    stlex ip, r1, [r3]
    cmp ip, #0
    bxeq lr
    b .L11
    \end{lstlisting}
\end{multicols}

Additionally, full memory barriers are required for earlier revisions as
described above:

\begin{multicols}{2}
    \begin{lstlisting}[style=arm]
xchg_w:
    movw r3, #:lower16:x
    movt r3, #:upper16:x
    dmb ish

    ldrex r2, [r3]
    cmp r2, r0
    it eq
    strexeq ip, r1, [r3]



    dmb ish
    bx lr
    \end{lstlisting}
    \columnbreak
    \begin{lstlisting}[style=arm]
xchg_s:
    movw r3, #:lower16:x
    movt r3, #:upper16:x
    dmb ish
.L11:
    ldrex r2, [r3]
    cmp r2, r0
    bne .L12
    strex ip, r1, [r3]
    cmp ip, #0
    bne .L11
.L12:
    dmb ish
    bx lr
    \end{lstlisting}
\end{multicols}

\subsubsection{PowerPC}

PowerPC also has a relaxed memory model and, similarly to ARMv7, requires memory
barriers to implement acquire/release operations.

\begin{multicols}{3}
    \begin{lstlisting}[style=powerpc,xleftmargin=0px,xrightmargin=0px]
load_rlx:
    addis 9,2,.LC0@toc@ha
    ld 9,.LC0@toc@l(9)

    lwz 3,0(9)



    extsw 3,3
    blr
    \end{lstlisting}
    \columnbreak
    \begin{lstlisting}[style=powerpc,xleftmargin=0px,xrightmargin=0px]
load_acq:
    addis 9,2,.LC1@toc@ha
    ld 9,.LC1@toc@l(9)

    lwz 3,0(9)
    cmpw 0,3,3
    bne- 0,$+4
    isync
    extsw 3,3
    blr
    \end{lstlisting}
    \columnbreak
    \begin{lstlisting}[style=powerpc,xleftmargin=0px,xrightmargin=0px]
load:
    addis 9,2,.LC2@toc@ha
    ld 9,.LC2@toc@l(9)
    sync
    lwz 3,0(9)
    cmpw 0,3,3
    bne- 0,$+4
    isync
    extsw 3,3
    blr
    \end{lstlisting}
\end{multicols}

\begin{multicols}{3}
    \begin{lstlisting}[style=powerpc,xleftmargin=0px,xrightmargin=0px]
store_rlx:
    addis 9,2,.LC3@toc@ha
    ld 9,.LC3@toc@l(9)

    stw 3,0(9)
    blr
    \end{lstlisting}
    \columnbreak
    \begin{lstlisting}[style=powerpc,xleftmargin=0px,xrightmargin=0px]
store_rel:
    addis 9,2,.LC4@toc@ha
    ld 9,.LC4@toc@l(9)
    lwsync
    stw 3,0(9)
    blr
    \end{lstlisting}
    \columnbreak
    \begin{lstlisting}[style=powerpc,xleftmargin=0px,xrightmargin=0px]
store:
    addis 9,2,.LC5@toc@ha
    ld 9,.LC5@toc@l(9)
    sync
    stw 3,0(9)
    blr
    \end{lstlisting}
\end{multicols}

\begin{multicols}{2}
    \begin{lstlisting}[style=powerpc]
xchg_w:
    addis 9,2,.LC5@toc@ha
    ld 10,.LC5@toc@l(9)
    sync

    lwarx 9,0,10
    cmpw 0,9,3
    bne 0,.L8
    stwcx. 4,0,10

.L8:
    isync
    blr
    \end{lstlisting}
    \columnbreak
    \begin{lstlisting}[style=powerpc]
xchg_s:
    addis 9,2,.LC6@toc@ha
    ld 10,.LC6@toc@l(9)
    sync
.L10:
    lwarx 9,0,10
    cmpw 0,9,3
    bne 0,.L11
    stwcx. 4,0,10
    bne 0,.L10
.L11:
    isync
    blr
    \end{lstlisting}
\end{multicols}
