\section{Binary search}
\label{sec:algo:bsearch}

Even though it may seem that examining every item in an array is required to
determine whether an element is present in it, the process can be significantly
improved, provided that the sequence is sorted in advance.  In that case, vast
portions of it can be eliminated with each comparison, a technique generally
known as \textit{divide-and-conquer}.  This provides a great advantage in cases
where searches are much more common than modifications to the sequence of items
--- i.e. where the cost of sorting the sequence can be paid once and the result
used by a number of searches.

The critical aspect of the implementation is to examine the \emph{middle}
element of the sequence.  If it is \emph{less} than the desired value, the
entire lower half of the array can be ignored.  Conversely, if it is
\emph{greater} than the desired value, the entire upper half can be ignored.
The process is then repeated until the result is found or the range becomes
empty.

Due to this process of halving in each iteration, this algorithm is called a
\textit{binary search}, and is a dramatic improvement over a linear search.  For
a sequence of $2^n$ elements, a linear search has to examine at worst all $2^n$
elements.  A binary search will reduce the search space to $2^{n-1}$ after the
first iteration, $2^{n-2}$ after the second, and so on, until it reaches $2^0 =
1$, resulting in only $n+1$ iterations\footnotemark.

\footnotetext{
    Section \secref{sec:algo:comp} explores the implications of this difference
    in much more detail.}

Figure \ref{fig:algo:bsearch0} shows one execution of a binary search on the
array \texttt{[0, 1, 2, 3, 5, 6, 7, 8]} for the value \texttt{3}.  The first
middle value is \texttt{5}, which is greater than the desired value, so the
upper portion of the array is eliminated.  The next middle value is \texttt{2},
which is less than the desired value, so the lower portion of the remaining
range is eliminated.  The last middle value is \texttt{3}, which is the desired
value, so the search ends.  Figure \ref{fig:algo:bsearch1} shows the search on
the same array for the value \texttt{4}, which is similar up to the last step,
where \texttt{x} is greater than \texttt{*m}, so the range becomes empty.

\begin{figure}[p]
    \centering
    \lstinputlisting[
        style=c,
        firstline=8,
        caption={Binary search},
        label={lst:algo:bsearch},
    ]{algo/bsearch.c}
    \vspace{2\baselineskip}
    \begin{subfigure}[h]{\textwidth}
        \centering
        \begin{subfigure}[h]{0.3\textwidth}
    \begin{tikzpicture}[>={Stealth[round]}, text height = 0.75em]
        \matrix {
            \node (n0) [rectangle, draw] {0}; &
            \node (n1) [rectangle, draw] {1}; &
            \node (n2) [rectangle, draw] {2}; &
            \node (n3) [rectangle, draw] {3}; &
            \node (n4) [rectangle, draw] {5}; &
            \node (n5) [rectangle, draw] {6}; &
            \node (n6) [rectangle, draw] {7}; &
            \node (n7) [rectangle, draw] {8}; \\
        };
        \node (b) at ($ (n0.south west) - (0, 0.75) $) {b};
        \node (m) at ($ (n4.south west) - (0, 0.75) $) {m};
        \node (e) at ($ (n7.south east) - (0, 0.75) $) {e};
        \draw[->] (b) -- ($ (n0.south west) - (0, 0.1) $);
        \draw[->] (m) -- ($ (n4.south west) - (0, 0.1) $);
        \draw[->] (e) -- ($ (n7.south east) - (0, 0.1) $);
    \end{tikzpicture}
\end{subfigure}
\quad
\begin{subfigure}[h]{0.3\textwidth}
    \begin{tikzpicture}[>={Stealth[round]}, text height = 0.75em]
        \matrix {
            \node (n0) [rectangle, draw] {0}; &
            \node (n1) [rectangle, draw] {1}; &
            \node (n2) [rectangle, draw] {2}; &
            \node (n3) [rectangle, draw] {3}; &
            \node (n4) [rectangle, draw] {5}; &
            \node (n5) [rectangle, draw] {6}; &
            \node (n6) [rectangle, draw] {7}; &
            \node (n7) [rectangle, draw] {8}; \\
        };
        \node (b) at ($ (n0.south west) - (0, 0.75) $) {b};
        \node (m) at ($ (n2.south west) - (0, 0.75) $) {m};
        \node (e) at ($ (n4.south west) - (0, 0.75) $) {e};
        \draw[->] (b) -- ($ (n0.south west) - (0, 0.1) $);
        \draw[->] (m) -- ($ (n2.south west) - (0, 0.1) $);
        \draw[->] (e) -- ($ (n4.south west) - (0, 0.1) $);
    \end{tikzpicture}
\end{subfigure}
\quad
\begin{subfigure}[h]{0.3\textwidth}
    \begin{tikzpicture}[>={Stealth[round]}, text height = 0.75em]
        \matrix {
            \node (n0) [rectangle, draw] {0}; &
            \node (n1) [rectangle, draw] {1}; &
            \node (n2) [rectangle, draw] {2}; &
            \node (n3) [rectangle, draw] {3}; &
            \node (n4) [rectangle, draw] {5}; &
            \node (n5) [rectangle, draw] {6}; &
            \node (n6) [rectangle, draw] {7}; &
            \node (n7) [rectangle, draw] {8}; \\
        };
        \node (b) at ($ (n3.south west) - (0, 0.75) $) {b};
        \node (e) at ($ (n4.south west) - (0, 0.75) $) {e};
        \draw[->] (b) -- ($ (n3.south west) - (0, 0.1) $);
        \draw[->] (e) -- ($ (n4.south west) - (0, 0.1) $);
    \end{tikzpicture}
\end{subfigure}

        \caption{\texttt{bsearch(3, b, e)}}
        \label{fig:algo:bsearch0}
    \end{subfigure}
    \\[\baselineskip]
    \begin{subfigure}[h]{\textwidth}
        \centering
        \begin{subfigure}[h]{0.3\textwidth}
    \begin{tikzpicture}[>={Stealth[round]}, text height = 0.75em]
        \matrix {
            \node (n0) [rectangle, draw] {0}; &
            \node (n1) [rectangle, draw] {1}; &
            \node (n2) [rectangle, draw] {2}; &
            \node (n3) [rectangle, draw] {3}; &
            \node (n4) [rectangle, draw] {5}; &
            \node (n5) [rectangle, draw] {6}; &
            \node (n6) [rectangle, draw] {7}; &
            \node (n7) [rectangle, draw] {8}; \\
        };
        \node (b) at ($ (n0.south west) - (0, 0.75) $) {b};
        \node (m) at ($ (n4.south west) - (0, 0.75) $) {m};
        \node (e) at ($ (n7.south east) - (0, 0.75) $) {e};
        \draw[->] (b) -- ($ (n0.south west) - (0, 0.1) $);
        \draw[->] (m) -- ($ (n4.south west) - (0, 0.1) $);
        \draw[->] (e) -- ($ (n7.south east) - (0, 0.1) $);
    \end{tikzpicture}
\end{subfigure}
\ldots \qquad
\begin{subfigure}[h]{0.3\textwidth}
    \begin{tikzpicture}[>={Stealth[round]}, text height = 0.75em]
        \matrix {
            \node (n0) [rectangle, draw] {0}; &
            \node (n1) [rectangle, draw] {1}; &
            \node (n2) [rectangle, draw] {2}; &
            \node (n3) [rectangle, draw] {3}; &
            \node (n4) [rectangle, draw] {5}; &
            \node (n5) [rectangle, draw] {6}; &
            \node (n6) [rectangle, draw] {7}; &
            \node (n7) [rectangle, draw] {8}; \\
        };
        \node (b) at ($ (n3.south west) - (0.1, 0.8) $) {b/m};
        \node (e) at ($ (n4.south west) - (0, 0.75) $) {e};
        \draw[->]
            ($ (b.north) + (0.1, 0) $)
            -- ($ (n3.south west) - (0, 0.1) $);
        \draw[->] (e) -- ($ (n4.south west) - (0, 0.1) $);
    \end{tikzpicture}
\end{subfigure}
\begin{subfigure}[h]{0.3\textwidth}
    \begin{tikzpicture}[>={Stealth[round]}, text height = 0.75em]
        \matrix {
            \node (n0) [rectangle, draw] {0}; &
            \node (n1) [rectangle, draw] {1}; &
            \node (n2) [rectangle, draw] {2}; &
            \node (n3) [rectangle, draw] {3}; &
            \node (n4) [rectangle, draw] {5}; &
            \node (n5) [rectangle, draw] {6}; &
            \node (n6) [rectangle, draw] {7}; &
            \node (n7) [rectangle, draw] {8}; \\
        };
        \node (b) at ($ (n4.south west) - (0, 0.75) $) {b/e};
        \draw[->] (b) -- ($ (n4.south west) - (0, 0.1) $);
    \end{tikzpicture}
\end{subfigure}

        \caption{\texttt{bsearch(4, b, e)}}
        \label{fig:algo:bsearch1}
    \end{subfigure}
    \caption{Binary search}
\end{figure}

Listing \ref{lst:algo:bsearch} shows the complete implementation.  It starts by
establishing the function preconditions and the loop termination condition.  The
inputs will be the value \texttt{x} to search for and the half-open interval
$\texttt{[b, e)}$ delimiting the sorted range.  The terminating condition as
described previously is that the range being examined becomes empty.  The
assertions after the loop are the same as the ones inside it, as will be
described next.

\lstinputlisting
    [style=c,linerange=8-11,belowskip=0pt]
    {algo/bsearch.c}
\begin{lstlisting}[aboveskip=0pt,belowskip=0pt]
        // ...
\end{lstlisting}
\lstinputlisting
    [style=c,linerange=24-30,aboveskip=0pt]
    {algo/bsearch.c}

The loop body starts with the invariants:

\lstinputlisting
    [style=c,linerange=12-15]
    {algo/bsearch.c}

\vspace{-\baselineskip}
\begin{itemize}
    \item
        Any reduced range should be within the bounds of the original and in the
        correct sequence.
    \item
        As the input range is repeatedly reduced, we guarantee \texttt{x} is not
        in any of the eliminated regions --- those now outside $\texttt{[b, e)}$.
    \item
        As \texttt{b} is moved right, it should always delimit elements that are
        less than \texttt{x}.
    \item
        Similarly, as \texttt{e} is moved left, it should always delimit
        elements that are greater than \texttt{x}.
\end{itemize}

From these and the loop termination condition that \texttt{b == e}, the
following inferences can be made about the value of \texttt{b} at the bottom of
the function:

\begin{itemize}
    \item
        $ib = ie \ \implies \ x \notin [ib, ie)$
        \begin{itemize}
            \item \texttt{x} cannot be in the range since it is empty
        \end{itemize}
    \item
        $b = ib \ \implies \ x \notin [ib, ie)$
        \begin{itemize}
            \item $x < *e \ \implies \ x < *ib$
            \item
                \texttt{x} cannot be in a sorted range if it is less than the
                first element
        \end{itemize}
    \item
        $e = ie \ \implies \ x \notin [ib, ie)$
        \begin{itemize}
            \item $b[-1] < x \ \implies e[-1] < x$
            \item
                \texttt{x} cannot be in a sorted range if the last element is
                less than it
        \end{itemize}
    \item
        otherwise $\implies \ x \notin [ib, ie)$
        \begin{itemize}
            \item $x < *e \ \implies \ x < *b$
            \item $b[-1] < x$
            \item
                \texttt{x} cannot be in a sorted range if $b[-1] < x$ and $x <
                b[0]$
        \end{itemize}
\end{itemize}

That is, as long as the invariants are maintained, exiting the loop means
\texttt{x} is not contained in the range.  This is reflected in the assertions
at the bottom of the function.  The implementation of the loop begins by
determining the middle element of the range.  This is done by computing its
length and using integer division to halve it while rounding down.\footnotemark.

\footnotetext{
    The difference is first converted to an unsigned value, since we know it is
    positive, so that the division by a power of two can be translated to a few
    simple bitwise operations.  Integer division rounds values towards zero,
    while division using shifts rounds towards negative infinity.  Converting a
    value to unsigned eliminates special cases in the generated machine code
    since those two cases have the same outcome for positive values.}

\lstinputlisting[style=c,linerange=16-17]{algo/bsearch.c}

Next, the middle value is compared to \texttt{x}: if it is equal, the value has
been found\footnotemark.  Otherwise, the search range is modified according to
the rules described previously.

\footnotetext{See exercise \ref{ex:algo:bsearch_bound}}

\lstinputlisting[style=c,linerange=18-23]{algo/bsearch.c}

We can observe that the invariants are preserved in both branches that advance
to the next iteration:

\begin{multicols}{2}
    \texttt{*m < x}
    \begin{align*}
        ib \leq b' & \impliedby
        \begin{cases}
            b' = m \\
            ib \leq b \\
            b \leq m \\
        \end{cases}
        \\
        b' \leq ie & \impliedby
        \begin{cases}
            b' = m + 1 \\
            m  < e \\
            e \leq ie \\
        \end{cases}
        \\
        b'[-1] < x & \impliedby
        \begin{cases}
            b' = m + 1 \\
            *m < x \\
        \end{cases}
    \end{align*}
    \\
    \columnbreak
    \\
    \texttt{*m > x}
    \begin{align*}
        ib \leq e' & \impliedby
        \begin{cases}
            e' = m \\
            ib \leq b \\
            b \leq m \\
        \end{cases}
        \\
        e' \leq ie & \impliedby
        \begin{cases}
            e' = m \\
            m < e \\
        \end{cases}
        \\
        x < *e' & \impliedby
        \begin{cases}
            e' = m \\
            *m \neq x \\
            x < *m \\
        \end{cases}
    \end{align*}
\end{multicols}

\subsection{Exercises}

\begin{enumerate}[label*=\arabic*.]
    \item
        \label{ex:algo:bsearch_middle}
        Prove that the expression \texttt{m = b + (e - b) / 2} correctly
        calculates the middle element of a range.
    \item
        \label{ex:algo:bsearch_overflow}
        A more common mathematical definition of the middle element would be
        \texttt{m = (b + e) / 2}.  Could that be used in our implementation of
        binary search?  Prove the answer and give examples.
    \item
        \label{ex:algo:bsearch_bound}
        Our version of binary search works and is significantly faster than
        linear search, but it performs two comparisons per iteration.  A
        closely-related fact is that it will return on the first element that is
        found to be equal to the input.

        Often, the range can have duplicate values and it is desired to find
        either the first position where a value occurs or the position after its
        last occurrence, e.g. to find the place where a new element should be
        inserted to keep the range sorted.  These two algorithms are called
        lower and upper bound.  More specifically, lower bound returns the
        position that partitions the range into elements that are \texttt{< x},
        while upper bound returns the position that partitions the range into
        elements that are \texttt{> v}.  For the array \texttt{[0, 1, 3, 3, 3,
        5, 6]}, the lower bound for value \texttt{3} is position \texttt{2},
        while the upper bound is position \texttt{5}.  These are the same
        results for the lower bound for value \texttt{2} and the upper bound for
        value \texttt{4}.

        Implement both versions and then rewrite the \texttt{binary\_search}
        function in terms of \texttt{lower\-\_bound}, all with suitable
        invariants.

        \begin{lstlisting}[style=c]
        int *lower_bound(int x, int *b, int *e);
        int *upper_bound(int x, int *b, int *e);
        \end{lstlisting}
\end{enumerate}
