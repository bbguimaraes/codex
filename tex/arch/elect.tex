\section{Electronics}

\begin{wrapfigure}{r}{0.3\textwidth}
    \centering
    \begin{tikzpicture}[
    >={Stealth[round]},
    inner xsep = 1em,
    minimum height = 2em,
]
    \node[circle, draw, minimum size = 4em] {};
    \draw[line width = 0.25em] (-0.125, -0.4) -- (-0.125, 0.4);
    \node (source) at (1.4142em, 1.75) {source};
    \node (drain) at (1.4142em, -1.75) {drain};
    \node (gate) at (-2, 0) {gate};
    \draw[->] (gate) -- (-0.125, 0);
    \draw[->] (source) -- (1.4142em, 1.4142em) -- (-0.125, 0.1);
    \draw[->] (-0.125, -0.1) -- (1.4142em, -1.4142em) -- (drain);
\end{tikzpicture}

    \caption{Transistor}
    \label{fig:arch:trans}
\end{wrapfigure}

The elemental component of digital circuits is the \textit{transistor}: a very
small semiconductor device which can be used to control electrical signals.
Transistors, as commonly used in circuits, have conceptually three
\textit{terminals}: \textit{source} and \textit{drain} (i.e. input/output) and a
\textit{gate} (figure \ref{fig:arch:trans})\footnotemark.  Whenever voltage is
applied to the gate, current is allowed to flow from the source to the drain.
In the absence of that voltage, no current flows through the conductor.  This
allows current to be applied selectively according to an input signal.  Multiple
transistors can be combined to create basic logic gates (figure
\ref{fig:arch:trans_gates}):

\footnotetext{
    Alternative names for these terminals are \textit{emitter},
    \textit{collector}, and \textit{base}, respectively.}

\begin{itemize}
    \item
        An \textit{and} gate is formed with two transistors whose signals are
        connected serially: current flows only if both inputs are enabled.
    \item
        An \textit{or} gate is formed with two transistors whose signals are
        connected in parallel: current flows when either of the inputs is
        enabled.
\end{itemize}

To simplify notation, these circuits are normally idealized in an abstract form.
A special symbol is used instead of the full circuit and only the transistor
gates and the output are shown, usually labeled \texttt{a}, \texttt{b}, and
\texttt{out}, respectivelly (figure \ref{fig:arch:gate_symbols}).

\begin{figure}[ht]
    \centering
    \begin{subfigure}{0.4\textwidth}
        \centering
        \begin{tikzpicture}[circuit_diagram]
            \node[and gate] (g) {};
            \node[left = 0.5 of g.input 1] (a) {\texttt{a}};
            \node[left = 0.5 of g.input 2] (b) {\texttt{b}};
            \node[right = 0.5 of g.output] (out) {out};
            \draw (g.input 1) -- (a);
            \draw (g.input 2) -- (b);
            \draw (g.output) -- (out);
        \end{tikzpicture}
        \caption{\textit{and} gate}
    \end{subfigure}
    \begin{subfigure}{0.4\textwidth}
        \centering
        \begin{tikzpicture}[circuit_diagram]
            \node[or gate] (g) {};
            \node[left = 0.5 of g.input 1] (a) {\texttt{a}};
            \node[left = 0.5 of g.input 2] (b) {\texttt{b}};
            \node[right = 0.5 of g.output] (out) {\texttt{out}};
            \draw (g.input 1) -- (a);
            \draw (g.input 2) -- (b);
            \draw (g.output) -- (out);
        \end{tikzpicture}
        \caption{\textit{or} gate}
    \end{subfigure}
    \caption{Logic gate symbols}
    \label{fig:arch:gate_symbols}
\end{figure}

\begin{figure}[p]
    \centering
    \begin{subfigure}{\textwidth}
        \centering
        \begin{subfigure}{0.225\textwidth}
            \begin{tikzpicture}[
    >={Stealth[round]},
    inner xsep = 1em,
    minimum height = 2em,
]
    \begin{scope}[yshift = 4em]
        \node[circle, draw, minimum size = 4em] {};
        \draw[line width = 0.25em] (-0.125, -0.4) -- (-0.125, 0.4);
        \node (source0) at (1.4142em, 1.75) {};
        \node[minimum size = 0pt, inner sep = 0pt]
            (drain0) at (1.4142em, -1.75) {};
        \node (gate0) at (-2, 0) {};
        \draw[->] (gate0) -- (-0.125, 0);
        \draw[->, red, thick]
            (source0) -- (1.4142em, 1.4142em) -- (-0.125, 0.1);
        \draw[->] (-0.125, -0.1) -- (1.4142em, -1.4142em) -- (drain0);
    \end{scope}
    \begin{scope}[yshift = -4em]
        \node[circle, draw, minimum size = 4em] {};
        \draw[line width = 0.25em] (-0.125, -0.4) -- (-0.125, 0.4);
        \node (source1) at (1.4142em, 1.75) {};
        \node[minimum size = 0pt, inner sep = 0pt]
            (drain1) at (1.4142em, -1.75) {};
        \node (gate1) at (-2, 0) {};
        \draw[->] (gate1) -- (-0.125, 0);
        \draw[->] (source1) -- (1.4142em, 1.4142em) -- (-0.125, 0.1);
        \draw[->] (-0.125, -0.1) -- (1.4142em, -1.4142em) -- (drain1);
    \end{scope}
\end{tikzpicture}

        \end{subfigure}
        \begin{subfigure}{0.225\textwidth}
            \begin{tikzpicture}[
    >={Stealth[round]},
    inner xsep = 1em,
    minimum height = 2em,
]
    \begin{scope}[yshift = 4em]
        \node[circle, draw, minimum size = 4em] {};
        \draw[line width = 0.25em] (-0.125, -0.4) -- (-0.125, 0.4);
        \node (source0) at (1.4142em, 1.75) {};
        \node[minimum size = 0pt, inner sep = 0pt]
            (drain0) at (1.4142em, -1.75) {};
        \node (gate0) at (-2, 0) {};
        \draw[->, red, thick] (gate0) -- (-0.125, 0);
        \draw[->, red, thick]
            (source0) -- (1.4142em, 1.4142em) -- (-0.125, 0.1);
        \draw[->, red, thick]
            (-0.125, -0.1) -- (1.4142em, -1.4142em) -- (drain0);
    \end{scope}
    \begin{scope}[yshift = -4em]
        \node[circle, draw, minimum size = 4em] {};
        \draw[line width = 0.25em] (-0.125, -0.4) -- (-0.125, 0.4);
        \node (source1) at (1.4142em, 1.75) {};
        \node[minimum size = 0pt, inner sep = 0pt]
            (drain1) at (1.4142em, -1.75) {};
        \node (gate1) at (-2, 0) {};
        \draw[->] (gate1) -- (-0.125, 0);
        \draw[->, red, thick]
            (source1) -- (1.4142em, 1.4142em) -- (-0.125, 0.1);
        \draw[->] (-0.125, -0.1) -- (1.4142em, -1.4142em) -- (drain1);
    \end{scope}
\end{tikzpicture}

        \end{subfigure}
        \begin{subfigure}{0.225\textwidth}
            \begin{tikzpicture}[
    >={Stealth[round]},
    inner xsep = 1em,
    minimum height = 2em,
]
    \begin{scope}[yshift = 4em]
        \node[circle, draw, minimum size = 4em] {};
        \draw[line width = 0.25em] (-0.125, -0.4) -- (-0.125, 0.4);
        \node (source0) at (1.4142em, 1.75) {};
        \node[minimum size = 0pt, inner sep = 0pt]
            (drain0) at (1.4142em, -1.75) {};
        \node (gate0) at (-2, 0) {};
        \draw[->] (gate0) -- (-0.125, 0);
        \draw[->, red, thick]
            (source0) -- (1.4142em, 1.4142em) -- (-0.125, 0.1);
        \draw[->] (-0.125, -0.1) -- (1.4142em, -1.4142em) -- (drain0);
    \end{scope}
    \begin{scope}[yshift = -4em]
        \node[circle, draw, minimum size = 4em] {};
        \draw[line width = 0.25em] (-0.125, -0.4) -- (-0.125, 0.4);
        \node (source1) at (1.4142em, 1.75) {};
        \node[minimum size = 0pt, inner sep = 0pt]
            (drain1) at (1.4142em, -1.75) {};
        \node (gate1) at (-2, 0) {};
        \draw[->, red, thick] (gate1) -- (-0.125, 0);
        \draw[->] (source1) -- (1.4142em, 1.4142em) -- (-0.125, 0.1);
        \draw[->] (-0.125, -0.1) -- (1.4142em, -1.4142em) -- (drain1);
    \end{scope}
\end{tikzpicture}

        \end{subfigure}
        \begin{subfigure}{0.225\textwidth}
            \begin{tikzpicture}[
    >={Stealth[round]},
    inner xsep = 1em,
    minimum height = 2em,
]
    \begin{scope}[yshift = 4em]
        \node[circle, draw, minimum size = 4em] {};
        \draw[line width = 0.25em] (-0.125, -0.4) -- (-0.125, 0.4);
        \node (source0) at (1.4142em, 1.75) {};
        \node[minimum size = 0pt, inner sep = 0pt]
            (drain0) at (1.4142em, -1.75) {};
        \node (gate0) at (-2, 0) {};
        \draw[->, red, thick] (gate0) -- (-0.125, 0);
        \draw[->, red, thick]
            (source0) -- (1.4142em, 1.4142em) -- (-0.125, 0.1);
        \draw[->, red, thick]
            (-0.125, -0.1) -- (1.4142em, -1.4142em) -- (drain0);
    \end{scope}
    \begin{scope}[yshift = -4em]
        \node[circle, draw, minimum size = 4em] {};
        \draw[line width = 0.25em] (-0.125, -0.4) -- (-0.125, 0.4);
        \node (source1) at (1.4142em, 1.75) {};
        \node[minimum size = 0pt, inner sep = 0pt]
            (drain1) at (1.4142em, -1.75) {};
        \node (gate1) at (-2, 0) {};
        \draw[->, red, thick] (gate1) -- (-0.125, 0);
        \draw[->, red, thick]
            (source1) -- (1.4142em, 1.4142em) -- (-0.125, 0.1);
        \draw[->, red, thick]
            (-0.125, -0.1) -- (1.4142em, -1.4142em) -- (drain1);
    \end{scope}
\end{tikzpicture}

        \end{subfigure}
        \\[2em]
        \caption{\textit{and} gate}
    \end{subfigure}
    \\[2em]
    \begin{subfigure}{\textwidth}
        \centering
        \begin{subfigure}{0.4\textwidth}
            \begin{tikzpicture}[
    >={Stealth[round]},
    inner xsep = 1em,
    minimum height = 2em,
]
    \node[inner sep = 0pt, minimum size = 0pt] (source) at (0.5, 2) {};
    \node[inner sep = 0pt, minimum size = 0pt] (drain) at (0.5, -1.5) {};
    \begin{scope}[xshift = -3em]
        \node[circle, draw, minimum size = 4em] {};
        \draw[line width = 0.25em] (-0.125, -0.4) -- (-0.125, 0.4);
        \node[inner sep = 0pt, minimum size = 0pt]
            (source0) at (1.4142em, 1.5) {};
        \node (gate0) at (-1.5, 0) {};
        \draw[->] (gate0) -- (-0.125, 0);
        \draw[->, red, thick]
            (source) |- (source0) -- (1.4142em, 1.4142em) -- (-0.125, 0.1);
        \draw (-0.125, -0.1) -- (1.4142em, -1.4142em) |- (drain);
    \end{scope}
    \begin{scope}[xshift = 3em]
        \node[circle, draw, minimum size = 4em] {};
        \draw[line width = 0.25em] (-0.125, -0.4) -- (-0.125, 0.4);
        \node[inner sep = 0pt, minimum size = 0pt]
            (source1) at (1.4142em, 1.5) {};
        \node (gate1) at (-1.5, 0) {};
        \draw[->] (gate1) -- (-0.125, 0);
        \draw[->, red, thick]
            (source) |- (source1) -- (1.4142em, 1.4142em) -- (-0.125, 0.1);
        \draw (-0.125, -0.1) -- (1.4142em, -1.4142em) |- (drain);
    \end{scope}
    \draw[->] (drain) -- +(0, -0.5);
\end{tikzpicture}

        \end{subfigure}
        \begin{subfigure}{0.4\textwidth}
            \begin{tikzpicture}[
    >={Stealth[round]},
    inner xsep = 1em,
    minimum height = 2em,
]
    \node[inner sep = 0pt, minimum size = 0pt] (source) at (0.5, 2) {};
    \node[inner sep = 0pt, minimum size = 0pt] (drain) at (0.5, -1.5) {};
    \begin{scope}[xshift = -3em]
        \node[circle, draw, minimum size = 4em] {};
        \draw[line width = 0.25em] (-0.125, -0.4) -- (-0.125, 0.4);
        \node[inner sep = 0pt, minimum size = 0pt]
            (source0) at (1.4142em, 1.5) {};
        \node (gate0) at (-1.5, 0) {};
        \draw[->, red, thick] (gate0) -- (-0.125, 0);
        \draw[->, red, thick]
            (source) |- (source0) -- (1.4142em, 1.4142em) -- (-0.125, 0.1);
        \draw[red, thick] (-0.125, -0.1) -- (1.4142em, -1.4142em) |- (drain);
    \end{scope}
    \begin{scope}[xshift = 3em]
        \node[circle, draw, minimum size = 4em] {};
        \draw[line width = 0.25em] (-0.125, -0.4) -- (-0.125, 0.4);
        \node[inner sep = 0pt, minimum size = 0pt]
            (source1) at (1.4142em, 1.5) {};
        \node (gate1) at (-1.5, 0) {};
        \draw[->] (gate1) -- (-0.125, 0);
        \draw[->, red, thick]
            (source) |- (source1) -- (1.4142em, 1.4142em) -- (-0.125, 0.1);
        \draw (-0.125, -0.1) -- (1.4142em, -1.4142em) |- (drain);
    \end{scope}
    \draw[->, red, thick] (drain) -- +(0, -0.5);
\end{tikzpicture}

        \end{subfigure}
        \\~\\
        \begin{subfigure}{0.4\textwidth}
            \begin{tikzpicture}[
    >={Stealth[round]},
    inner xsep = 1em,
    minimum height = 2em,
]
    \node[inner sep = 0pt, minimum size = 0pt] (source) at (0.5, 2) {};
    \node[inner sep = 0pt, minimum size = 0pt] (drain) at (0.5, -1.5) {};
    \begin{scope}[xshift = -3em]
        \node[circle, draw, minimum size = 4em] {};
        \draw[line width = 0.25em] (-0.125, -0.4) -- (-0.125, 0.4);
        \node[inner sep = 0pt, minimum size = 0pt]
            (source0) at (1.4142em, 1.5) {};
        \node (gate0) at (-1.5, 0) {};
        \draw[->] (gate0) -- (-0.125, 0);
        \draw[->, red, thick]
            (source) |- (source0) -- (1.4142em, 1.4142em) -- (-0.125, 0.1);
        \draw (-0.125, -0.1) -- (1.4142em, -1.4142em) |- (drain);
    \end{scope}
    \begin{scope}[xshift = 3em]
        \node[circle, draw, minimum size = 4em] {};
        \draw[line width = 0.25em] (-0.125, -0.4) -- (-0.125, 0.4);
        \node[inner sep = 0pt, minimum size = 0pt]
            (source1) at (1.4142em, 1.5) {};
        \node (gate1) at (-1.5, 0) {};
        \draw[->, red, thick] (gate1) -- (-0.125, 0);
        \draw[->, red, thick]
            (source) |- (source1) -- (1.4142em, 1.4142em) -- (-0.125, 0.1);
        \draw[red, thick] (-0.125, -0.1) -- (1.4142em, -1.4142em) |- (drain);
    \end{scope}
    \draw[->, red, thick] (drain) -- +(0, -0.5);
\end{tikzpicture}

        \end{subfigure}
        \begin{subfigure}{0.4\textwidth}
            \begin{tikzpicture}[
    >={Stealth[round]},
    inner xsep = 1em,
    minimum height = 2em,
]
    \node[inner sep = 0pt, minimum size = 0pt] (source) at (0.5, 2) {};
    \node[inner sep = 0pt, minimum size = 0pt] (drain) at (0.5, -1.5) {};
    \begin{scope}[xshift = -3em]
        \node[circle, draw, minimum size = 4em] {};
        \draw[line width = 0.25em] (-0.125, -0.4) -- (-0.125, 0.4);
        \node[inner sep = 0pt, minimum size = 0pt]
            (source0) at (1.4142em, 1.5) {};
        \node (gate0) at (-1.5, 0) {};
        \draw[->, red, thick] (gate0) -- (-0.125, 0);
        \draw[->, red, thick]
            (source) |- (source0) -- (1.4142em, 1.4142em) -- (-0.125, 0.1);
        \draw[red, thick] (-0.125, -0.1) -- (1.4142em, -1.4142em) |- (drain);
    \end{scope}
    \begin{scope}[xshift = 3em]
        \node[circle, draw, minimum size = 4em] {};
        \draw[line width = 0.25em] (-0.125, -0.4) -- (-0.125, 0.4);
        \node[inner sep = 0pt, minimum size = 0pt]
            (source1) at (1.4142em, 1.5) {};
        \node (gate1) at (-1.5, 0) {};
        \draw[->, red, thick] (gate1) -- (-0.125, 0);
        \draw[->, red, thick]
            (source) |- (source1) -- (1.4142em, 1.4142em) -- (-0.125, 0.1);
        \draw[red, thick] (-0.125, -0.1) -- (1.4142em, -1.4142em) |- (drain);
    \end{scope}
    \draw[->, red, thick] (drain) -- +(0, -0.5);
\end{tikzpicture}

        \end{subfigure}
        \\[1em]
        \caption{\textit{or} gate}
    \end{subfigure}
    \caption{Logic gates}
    \label{fig:arch:trans_gates}
\end{figure}
