\section{Semaphores}

\subsection{Mutex}

\label{subsec:conc:mutex}

\subsection{\texttt{futex(2)}}

A \textit{futex} (short for \textit{fast user-space mutex}) is a primitive in
the Linux kernel used to build higher-level concurrency primitives, introduced
in 2003 in the 2.6 stable series by Hubertus Franke, Matthew Kirkwood, Ingo
Molnár, and Rusty Russell.  The ``fast'' portion of the name refers to its
user-space component, which can be as simple as a single atomic operation on an
integer.  The \texttt{futex(2)} system call handles the slow case of suspending
threads when there the lock is contended.  Futexes are fast because the cost of
a system call can be completely avoided when there is no
contention.\footnotemark

\footnotetext{
    V. section \secref{sec:concurrency:rcu} for another example of optimizing
    for the uncontended case.}

The wait queue of suspended threads exists on the kernel side and is associated
with the memory address supplied by user space.  The \texttt{futex(2)} system
call is a multiplexer (in the same manner as \texttt{ioctl(2)}) whose two main
operations are \texttt{FUTEX\_WAIT} and \texttt{FUTEX\_WAKE}.  \texttt{WAIT}
adds the current thread to the wait queue (after ensuring --- atomically ---
that the value being waited on has not changed), while \texttt{WAKE} resumes
threads currently on the queue.  Other operations are available as more
specialized versions of \texttt{WAKE}:

\begin{itemize}
    \item \texttt{FUTEX\_CMP\_REQUEUE}:
        wakes a specified number of threads and transfers any remaining ones to
        a second queue (avoids a \textit{thundering herd} on wake)
    \item \texttt{FUTEX\_WAKE\_OP}:
        provides a more flexible waking comparison operation based on a number
        of available operations on a second value
\end{itemize}

Higher-level concurrency primitives are built using these two operations.  One
of the simplest is an event that can be signaled and waited on (extracted,
slightly reformatted, from \cite{Drepper2011}, v. for further examples and
discussion):

\begin{lstlisting}[style=c++]
class event {
public:
    void signal(void) { futex_wake(&++this->val, INT_MAX); }
    void wait(void) { futex_wait(&this->val, this->val); }
private:
    int val = 0;
};
\end{lstlisting}
