\section{Linux}

\subsection{Capabilities}

Traditionally, a \textit{privileged} process is Unix systems is one with
effective UID of 0.  Many system administrator operations in the kernel can only
be done by this type of process.  While very simple to implement, this scheme
has the disadvantage that the security model is binary: a process has either
\emph{no} or \emph{all} privileges.

\textit{Capabilities} are an improvement present in Linux and some other Unix
derivatives.  Initially part of the unsuccessful POSIX 1003.1e draft standard,
capabilities break the single "privileged" status of a process\footnotemark into
many parts (v. \texttt{capabilities(7)} for the full list).  The kernel then
checks for a specific capability instead of an UID; a privileged process can
drop all capabilities it does not need and greatly reduce the possibility that
it could be used as an attack vector in a compromised system.

\footnotetext{
    Throughout this section, "process" is used for simplicity, but note that
    capabilities are a thread (i.e. task) attribute.}

As a simple example, a program which sets the system time, such as \texttt{date
--set …}, needs only the \ident{CAP_SYS_TIME} capability.  On a traditional Unix
system without capabilities, that program would need to have an effective UID of
0, meaning it could in principle perform \emph{any} privileged operation.  There
is no reason for it to be able to change mount points, network configuration, or
perform any of the many other system administration operations: capabilities
allow a process to restrict its privileges to just those necessary for its
function.

While capabilities have been referred to as "a set" of "a process", that is a
simplification of the actual Linux implementation.  They can be assigned to
processes and files, and there is not one but four sets.

\subsubsection{Sets}

Ultimately, capabilities are sets of 64 bits associated with each process or
file.  Three sets are associated with each, while two other are specific to
processes:

\begin{itemize}
    \item
        The \textit{effective} set is the one used by the kernel in its
        permission checks.
    \item
        The \textit{permitted} set serves, in conjunction with the effective
        set, an analogous function to the real/effective UIDs: a process can
        temporarily reduce its effective (and inheritable) set and later restore
        it to any subset of its permitted set.  Removing a capability from this
        set is irreversible (by the process, but see capabilities attached to
        files below).
    \item
        The \textit{inheritable} set determines which capabilities are preserved
        by a privileged process after an \texttt{exec} system call\footnotemark.
        Only privileged processes (\ident{CAP_SETPCAP}) can add capabilities to
        this set.
    \item
        The \textit{ambient} set determines which capabilities are preserved by
        an unprivileged process after an \texttt{exec} system call.  It is a
        subset of the intersection of the permitted and inheritable sets.
    \item
        The \textit{bounding} set ultimately limits all capabilities which a
        process can acquire.  This set is maintained when an \texttt{exec}
        system call is performed.  The \texttt{init} process starts with a full
        set, and at any time a privileged process (\ident{CAP_SETPCAP}) can
        irreversibly remove a capability from its set --- and from all of its
        descendants as a consequence.
\end{itemize}

\footnotetext{
    Capabilities are not preserved across \texttt{exec} calls because the call
    may fail and may require privileges that are not required by the new
    executable (e.g. \ident{CAP_DAC_OVERRIDE}).}

The sets associated with a process can be examined via the \texttt{proc} file
system:

\begin{lstlisting}
$ p=$(pgrep --full --oldest systemd-timesyncd)
$ grep Cap /proc/$p/status
CapInh: 0000000002000000
CapPrm: 0000000002000000
CapEff: 0000000002000000
CapBnd: 0000000002000000
CapAmb: 0000000002000000
$ capsh --decode=0000000002000000
0x0000000002000000=cap_sys_time
$ getpcaps $p
2461: cap_sys_time=eip
\end{lstlisting} %$

\subsubsection{Files}

Sets can be associated with files using \texttt{security} extended attributes to
grant new capabilities to processes when it is used in an \texttt{exec} system
call:

\begin{itemize}
    \item The \textit{permitted} set is used to grant new capabilities.
    \item
        The \textit{inheritable} set further limits a process' own inheritable
        set when the preserved capabilities are calculated.
    \item
        The \textit{effective} set similarly limits a process' own effective
        set.
\end{itemize}

In addition, the ambient set of a process is cleared if the executed file has
any of: set-user-ID, set-group-ID, or file capabilities.  The complete equations
for the new permitted ($P'_p$), effective ($P'_e$), inheritable ($P'_i$),
ambient ($P'_a$), and bounding ($P'_b$) sets based on the previous ($P$) and
file ($F$) sets are:

\begin{align*}
    P'_p &= (P_i \cap F_i) \cup (P_b \cap F_p) \cup P_a \\
    P'_a &= \texttt{!file\_privileged} \times P_a \\
    P'_e &= F_e\text{ ? }P'_p\text{ : }P'_a \\
    P'_i &= P_p \\
    P'_b &= P_b \\
\end{align*}

\subsubsection{UID transitions}

The following changes, designed to adhere to the traditional Unix semantics, are
made to the capability sets of a process when it changes UID:

\begin{itemize}
    \item
        If either the real, effective, or saved set-user IDs are zero but none
        are after the transition, the permitted, effective, and ambient sets are
        cleared.
    \item
        If the effective UID transitions from zero to non-zero, the effective
        set is cleared.  The reverse transition repopulates it using the values
        in the permitted set.
    \item
        If the file system UID transitions from zero to non-zero or the reverse,
        the same changes as the previous item are made, but restricted to a few
        file-system-related capabilities (v. \texttt{capabilities(7)}).
\end{itemize}

The first rule can be disabled using a mechanism called \texttt{securebits}
(also available via \texttt{libcap(3)})\footnotemark:

\footnotetext{
    V. \cite{Edge2015} for a discussion of the introduction of ambient
    capabilities to enable unprivileged processes to inherit capabilities.}

\begin{lstlisting}[style=c]
prctl(PR_SET_SECUREBITS, prctl(PR_GET_SECUREBITS) | SECBIT_KEEP_CAPS);
\end{lstlisting}

A privileged process can use this to execute a program using an unprivileged
user with a few select capabilities:

\begin{lstlisting}[style=c]
cap_set_secbits(cap_get_secbits() | SECBIT_KEEP_CAPS);
setresgid(100, 100, 100);
setresuid(1000, 1000, 1000);
cap_t c = cap_get_proc();
cap_clear(c);
const cap_value_t v = CAP_NET_ADMIN;
cap_set_flag(c, CAP_PERMITTED, 1, &v, CAP_SET);
cap_set_flag(c, CAP_INHERITABLE, 1, &v, CAP_SET);
cap_set_proc(c);
cap_free(c);
cap_set_ambient(CAP_NET_ADMIN, CAP_SET);
execv(/* ... */);
\end{lstlisting}
