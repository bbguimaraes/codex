\section{Sorting}

\subsection{Ordering}
\label{subsec:algo:ordering}

\subsection{Selection sort}

\begin{figure}[p]
    \centering
    \lstinputlisting[
        style=c++,
        firstline=7,
        caption={Selection sort},
        label={lst:algo:selection},
    ]{algo/sort/selection.cpp}
    \vspace{2\baselineskip}
    \begin{subfigure}[h]{0.3\textwidth}
    \begin{tikzpicture}[>={Stealth[round]}, text height = 0.75em]
        \matrix {
            \node (n0) [rectangle, draw] {2}; &
            \node (n1) [rectangle, draw] {1}; &
            \node (n2) [rectangle, draw] {7}; &
            \node (n3) [rectangle, draw] {5}; &
            \node (n4) [rectangle, draw] {6}; &
            \node (n5) [rectangle, draw] {0}; &
            \node (n6) [rectangle, draw] {4}; &
            \node (n7) [rectangle, draw] {3}; \\
        };
        \node (b) at ($ (n0.south west) - (0, 0.75) $) {b};
        \node (m) at ($ (n5.south west) - (0, 0.75) $) {min};
        \node (e) at ($ (n7.south east) - (0, 0.75) $) {e};
        \draw[->] (b) -- ($ (n0.south west) - (0, 0.1) $);
        \draw[->] (m) -- ($ (n5.south west) - (0, 0.1) $);
        \draw[->] (e) -- ($ (n7.south east) - (0, 0.1) $);
    \end{tikzpicture}
\end{subfigure}
\quad
\begin{subfigure}[h]{0.3\textwidth}
    \begin{tikzpicture}[>={Stealth[round]}, text height = 0.75em]
        \matrix {
            \node (n0) [rectangle, draw, fill = green!50] {0}; &
            \node (n1) [rectangle, draw] {1}; &
            \node (n2) [rectangle, draw] {7}; &
            \node (n3) [rectangle, draw] {5}; &
            \node (n4) [rectangle, draw] {6}; &
            \node (n5) [rectangle, draw] {2}; &
            \node (n6) [rectangle, draw] {4}; &
            \node (n7) [rectangle, draw] {3}; \\
        };
        \node (b) at ($ (n1.south west) - (0, 0.8) $) {b/min};
        \node (e) at ($ (n7.south east) - (0, 0.75) $) {e};
        \draw[->] (b) -- ($ (n1.south west) - (0, 0.1) $);
        \draw[->] (e) -- ($ (n7.south east) - (0, 0.1) $);
    \end{tikzpicture}
\end{subfigure}
\quad
\begin{subfigure}[h]{0.3\textwidth}
    \begin{tikzpicture}[>={Stealth[round]}, text height = 0.75em]
        \matrix {
            \node (n0) [rectangle, draw, fill = green!50] {0}; &
            \node (n1) [rectangle, draw, fill = green!50] {1}; &
            \node (n2) [rectangle, draw] {7}; &
            \node (n3) [rectangle, draw] {5}; &
            \node (n4) [rectangle, draw] {6}; &
            \node (n5) [rectangle, draw] {2}; &
            \node (n6) [rectangle, draw] {4}; &
            \node (n7) [rectangle, draw] {3}; \\
        };
        \node (b) at ($ (n2.south west) - (0, 0.75) $) {b};
        \node (m) at ($ (n5.south west) - (0, 0.75) $) {min};
        \node (e) at ($ (n7.south east) - (0, 0.75) $) {e};
        \draw[->] (b) -- ($ (n2.south west) - (0, 0.1) $);
        \draw[->] (m) -- ($ (n5.south west) - (0, 0.1) $);
        \draw[->] (e) -- ($ (n7.south east) - (0, 0.1) $);
    \end{tikzpicture}
\end{subfigure}
\\[\baselineskip]
\begin{subfigure}[h]{0.3\textwidth}
    \begin{tikzpicture}[>={Stealth[round]}, text height = 0.75em]
        \matrix {
            \node (n0) [rectangle, draw, fill = green!50] {0}; &
            \node (n1) [rectangle, draw, fill = green!50] {1}; &
            \node (n2) [rectangle, draw, fill = green!50] {2}; &
            \node (n3) [rectangle, draw] {5}; &
            \node (n4) [rectangle, draw] {6}; &
            \node (n5) [rectangle, draw] {7}; &
            \node (n6) [rectangle, draw] {4}; &
            \node (n7) [rectangle, draw] {3}; \\
        };
        \node (b) at ($ (n3.south west) - (0, 0.75) $) {b};
        \node (m) at ($ (n7.south west) - (0, 0.75) $) {min};
        \node (e) at ($ (n7.south east) - (0, 0.75) $) {e};
        \draw[->] (b) -- ($ (n3.south west) - (0, 0.1) $);
        \draw[->] (m) -- ($ (n7.south west) - (0, 0.1) $);
        \draw[->] (e) -- ($ (n7.south east) - (0, 0.1) $);
    \end{tikzpicture}
\end{subfigure}
\quad
\begin{subfigure}[h]{0.3\textwidth}
    \begin{tikzpicture}[>={Stealth[round]}, text height = 0.75em]
        \matrix {
            \node (n0) [rectangle, draw, fill = green!50] {0}; &
            \node (n1) [rectangle, draw, fill = green!50] {1}; &
            \node (n2) [rectangle, draw, fill = green!50] {2}; &
            \node (n3) [rectangle, draw, fill = green!50] {3}; &
            \node (n4) [rectangle, draw] {6}; &
            \node (n5) [rectangle, draw] {7}; &
            \node (n6) [rectangle, draw] {4}; &
            \node (n7) [rectangle, draw] {5}; \\
        };
        \node (b) at ($ (n4.south west) - (0, 0.75) $) {b};
        \node (m) at ($ (n6.south west) - (0, 0.75) $) {min};
        \node (e) at ($ (n7.south east) - (0, 0.75) $) {e};
        \draw[->] (b) -- ($ (n4.south west) - (0, 0.1) $);
        \draw[->] (m) -- ($ (n6.south west) - (0, 0.1) $);
        \draw[->] (e) -- ($ (n7.south east) - (0, 0.1) $);
    \end{tikzpicture}
\end{subfigure}
\quad
\begin{subfigure}[h]{0.3\textwidth}
    \begin{tikzpicture}[>={Stealth[round]}, text height = 0.75em]
        \matrix {
            \node (n0) [rectangle, draw, fill = green!50] {0}; &
            \node (n1) [rectangle, draw, fill = green!50] {1}; &
            \node (n2) [rectangle, draw, fill = green!50] {2}; &
            \node (n3) [rectangle, draw, fill = green!50] {3}; &
            \node (n4) [rectangle, draw, fill = green!50] {4}; &
            \node (n5) [rectangle, draw] {7}; &
            \node (n6) [rectangle, draw] {6}; &
            \node (n7) [rectangle, draw] {5}; \\
        };
        \node (b) at ($ (n5.south west) - (0, 0.75) $) {b};
        \node (m) at ($ (n7.south west) - (0, 0.75) $) {min};
        \node (e) at ($ (n7.south east) - (0, 0.75) $) {e};
        \draw[->] (b) -- ($ (n5.south west) - (0, 0.1) $);
        \draw[->] (m) -- ($ (n7.south west) - (0, 0.1) $);
        \draw[->] (e) -- ($ (n7.south east) - (0, 0.1) $);
    \end{tikzpicture}
\end{subfigure}
\\[\baselineskip]
\begin{subfigure}[h]{0.3\textwidth}
    \begin{tikzpicture}[>={Stealth[round]}, text height = 0.75em]
        \matrix {
            \node (n0) [rectangle, draw, fill = green!50] {0}; &
            \node (n1) [rectangle, draw, fill = green!50] {1}; &
            \node (n2) [rectangle, draw, fill = green!50] {2}; &
            \node (n3) [rectangle, draw, fill = green!50] {3}; &
            \node (n4) [rectangle, draw, fill = green!50] {4}; &
            \node (n5) [rectangle, draw, fill = green!50] {5}; &
            \node (n6) [rectangle, draw] {6}; &
            \node (n7) [rectangle, draw] {7}; \\
        };
        \node (b) at ($ (n6.south west) - (0, 0.8) $) {b/min};
        \node (e) at ($ (n7.south east) - (0, 0.75) $) {e};
        \draw[->] (b) -- ($ (n6.south west) - (0, 0.1) $);
        \draw[->] (e) -- ($ (n7.south east) - (0, 0.1) $);
    \end{tikzpicture}
\end{subfigure}
\quad
\begin{subfigure}[h]{0.3\textwidth}
    \begin{tikzpicture}[>={Stealth[round]}, text height = 0.75em]
        \matrix {
            \node (n0) [rectangle, draw, fill = green!50] {0}; &
            \node (n1) [rectangle, draw, fill = green!50] {1}; &
            \node (n2) [rectangle, draw, fill = green!50] {2}; &
            \node (n3) [rectangle, draw, fill = green!50] {3}; &
            \node (n4) [rectangle, draw, fill = green!50] {4}; &
            \node (n5) [rectangle, draw, fill = green!50] {5}; &
            \node (n6) [rectangle, draw, fill = green!50] {6}; &
            \node (n7) [rectangle, draw] {7}; \\
        };
        \node (b) at ($ (n7.south west) - (0.2, 0.8) $) {b/min};
        \node (e) at ($ (n7.south east) - (0, 0.75) $) {e};
        \draw[->]
            ($ (b.north) + (0.2, 0) $)
            -- ($ (n7.south west) - (0, 0.1) $);
        \draw[->] (e) -- ($ (n7.south east) - (0, 0.1) $);
    \end{tikzpicture}
\end{subfigure}
\quad
\begin{subfigure}[h]{0.3\textwidth}
    \begin{tikzpicture}[>={Stealth[round]}, text height = 0.75em]
        \matrix {
            \node (n0) [rectangle, draw, fill = green!50] {0}; &
            \node (n1) [rectangle, draw, fill = green!50] {1}; &
            \node (n2) [rectangle, draw, fill = green!50] {2}; &
            \node (n3) [rectangle, draw, fill = green!50] {3}; &
            \node (n4) [rectangle, draw, fill = green!50] {4}; &
            \node (n5) [rectangle, draw, fill = green!50] {5}; &
            \node (n6) [rectangle, draw, fill = green!50] {6}; &
            \node (n7) [rectangle, draw, fill = green!50] {7}; \\
        };
        \node (b) at ($ (n7.south east) - (0, 0.8) $) {b/e};
        \draw[->] (b) -- ($ (n7.south east) - (0, 0.1) $);
    \end{tikzpicture}
\end{subfigure}

    \caption{Selection sort}
    \label{fig:algo:selection}
\end{figure}

One of the simplest sorting algorithms is \textit{selection sort}, which works
by progressively sorting the left side of the range one element at a time
(figure \ref{fig:algo:selection}).  In each iteration, the minimum value in the
unsorted portion is found (\emph{selected}) and placed at the beginning,
increasing the size of the sorted range by one element.  When all positions of
the range have gone through this procedure, the range is sorted.

The implementation of the algorithm follows naturally from this description
(listing \ref{lst:algo:selection}).  In each iteration, the beginning of the
range is guaranteed to already be sorted.  The minimum element of the rest of
the unsorted range is found, swapped with the element at the beginning of the
range, guaranteeing that the range is now partitioned with respect to the
selected element.
