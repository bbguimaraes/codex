\paragraph{\ref{ch:struct}.\ref{ex:struct:list_reverse}}

There are several ways to conceptualize the operations involved in the reversal
of a list.  One of them is to start with an empty list and continuously move one
element from the front of the old list to the front of the new one.  Always
pushing to the front of the new list effectively reverses the order of nodes.

\begin{lstlisting}[style=c]
struct node *list_reverse(struct node *n) {
    struct node *ret = NULL;
    for(struct node *next = NULL; n; n = next) {
        next = n->next;
        list_push_front(&ret, n);
    }
    return ret;
}
\end{lstlisting}

\paragraph{\ref{ch:struct}.\ref{ex:struct:list_remove}}

This exercise shows the benefit of treating the list head and links
homogeneously.  The function is passed a double pointer, so it can examine the
value in the node via a double dereference, but also modify the head or node
link pointer through it.  Figures \ref{fig:sol:list_remove0} and
\ref{fig:sol:list_remove1} show both cases.

\begin{lstlisting}[style=c]
struct node *list_remove(struct node **l, int value) {
    for(; *l; l = &(*l)->next) {
        if((*l)->value != value)
            continue;
        struct node *ret = *l;
        *l = ret->next;
        return ret;
    }
    return NULL;
}
\end{lstlisting}

\begin{figure}[p]
    \centering
    \begin{tikzpicture}[
        every node/.style = {minimum height = 16px},
        code/.style = {align = center, font = \small},
        >={Stealth[round]},
    ]
    \node (obj0) [object = 2] at (0, 4.5) {
        \vrule height 0.5em depth 0.25em width 0pt
        \itshape obj
        \nodepart{two} \vrule height 0.5em depth 0.25em width 0pt
        head
    };
    \node (obj1) [object = 2] at (0, 0) {
        \vrule height 0.5em depth 0.25em width 0pt
        \itshape obj
        \nodepart{two} \vrule height 0.5em depth 0.25em width 0pt
        head
    };
    \node (n0_0) [lnode, right = of obj0.two east] {0};
    \node (n1_0) [lnode, right = of n0_0] {1};
    \node (n2_0) [lnode, right = of n1_0] {2};
    \node (n0_1) [lnode, right = of obj1.two east] {0};
    \node (n1_1) [lnode, right = of n0_1] {1};
    \node (n2_1) [lnode, right = of n1_1] {2};
    \node (l0) [left = of obj0.two west] {l};
    \node (l1) [left = of obj1.two west] {l};
    \node (t0) [code, xshift = 2.5em] at (-1.5, 2) {
        \texttt{(*l)->value == 0} \\
        \texttt{obj.head->value == 0} \\
        \texttt{(*obj.head).value == 0} \\
        \texttt{n0.value == 0}
    };
    \node (t1) [code, right = of t0] {
        \texttt{*l = (*l)->next} \\
        \texttt{obj.head = (*obj.head).next} \\
        \texttt{obj.head = n0.next} \\
        \texttt{obj.head = \&n1}
    };
    \node [below = of n0_0, yshift = 2.5em] {n0};
    \node [below = of n1_0, yshift = 2.5em] {n1};
    \node [below = of n2_0, yshift = 2.5em] {n2};
    \node [below = of n0_1, yshift = 2.5em] {n0};
    \node [below = of n1_1, yshift = 2.5em] {n1};
    \node [below = of n2_1, yshift = 2.5em] {n2};
    \draw[->] (t0) -- (t1);
    \draw[->] (n0_0) -- (n1_0);
    \draw[->] (n1_0) -- (n2_0);
    \draw[->] (n0_1) -- (n1_1);
    \draw[->] (n1_1) -- (n2_1);
    \draw[->] (l0) -- (obj0.two west);
    \draw[->] (l1) -- (obj1.two west);
    \draw[->] (obj0.two east) -- (n0_0);
    \draw[->]
        (obj1.two east)
        .. controls ($ (n0_1) + (0, 4em) $) and +(0, 0)
        .. (n1_1.west);
\end{tikzpicture}

    \caption{\texttt{list\_remove(\&l, 0)}}
    \label{fig:sol:list_remove0}
\end{figure}

\begin{figure}[p]
    \centering
    \begin{tikzpicture}[
        every node/.style={minimum height = 16px},
        code/.style = {align = center, font = \small},
        >={Stealth[round]},
    ]
    \node (obj0) [object = 2] at (0, 7.5) {
        \vrule height 0.5em depth 0.25em width 0pt
        \itshape obj
        \nodepart{two} \vrule height 0.5em depth 0.25em width 0pt
        head
    };
    \node (obj1) [object = 2] at (0, 4) {
        \vrule height 0.5em depth 0.25em width 0pt
        \itshape obj
        \nodepart{two} \vrule height 0.5em depth 0.25em width 0pt
        head
    };
    \node (obj2) [object = 2] at (0, 0) {
        \vrule height 0.5em depth 0.25em width 0pt
        \itshape obj
        \nodepart{two} \vrule height 0.5em depth 0.25em width 0pt
        head
    };
    \node (n0_0) [lnode, right = of obj0.two east] {0};
    \node (n1_0) [lnode, right = of n0_0] {1};
    \node (n2_0) [lnode, right = of n1_0] {2};
    \node (n0_1) [lnode, right = of obj1.two east] {0};
    \node (n1_1) [lnode, right = of n0_1] {1};
    \node (n2_1) [lnode, right = of n1_1] {2};
    \node (n0_2) [lnode, right = of obj2.two east] {0};
    \node (n1_2) [lnode, right = of n0_2] {1};
    \node (n2_2) [lnode, right = of n1_2] {2};
    \node (l0) [left = of obj0.two west] {l};
    \node (l1) [above = of n0_1.two north, yshift = -1.5em] {l};
    \node (l2) [above = of n0_2.two north, yshift = -1.5em] {l};
    \node (t0) [code] at (0, 5.5) {
        \texttt{(*l)->value != 1} \\
        \texttt{n0.value != 1}
    };
    \node (t1) [code, right = of t0] {
        \texttt{l = \&(*l)->next} \\
        \texttt{l = \&(*obj.head).next} \\
        \texttt{l = \&n0.next}
    };
    \node (t2) [code] at (0, 1.75) {
        \texttt{(*l)->value == 1} \\
        \texttt{n0.value == 1}
    };
    \node (t3) [code, right = of t2] {
        \texttt{*l = (*l)->next} \\
        \texttt{n0.next = (*obj.head).next} \\
        \texttt{n0.next = n1.next} \\
        \texttt{n0.next = \&n2}
    };
    \node [below = of n0_0, yshift = 2.5em] {n0};
    \node [below = of n1_0, yshift = 2.5em] {n1};
    \node [below = of n2_0, yshift = 2.5em] {n2};
    \node [below = of n0_1, yshift = 2.5em] {n0};
    \node [below = of n1_1, yshift = 2.5em] {n1};
    \node [below = of n2_1, yshift = 2.5em] {n2};
    \node [below = of n0_2, yshift = 2.5em] {n0};
    \node [below = of n1_2, yshift = 2.5em] {n1};
    \node [below = of n2_2, yshift = 2.5em] {n2};
    \draw[->] (t0) -- (t1);
    \draw[->] (t2) -- (t3);
    \draw[->] (n0_0) -- (n1_0);
    \draw[->] (n1_0) -- (n2_0);
    \draw[->] (n0_1) -- (n1_1);
    \draw[->] (n1_1) -- (n2_1);
    \draw[->] (n0_1) -- (n1_1);
    \draw[->] (n1_1) -- (n2_1);
    \draw[->]
        (n0_2.east)
        .. controls ($ (n1_2) + (0, 4em) $) and +(0, 0)
        .. (n2_2.west);
    \draw[->] (n1_2) -- (n2_2);
    \draw[->] (l0) -- (obj0.two west);
    \draw[->] (l1) -- (n0_1.two north);
    \draw[->] (l2) -- (n0_2.two north);
    \draw[->] (obj0.two east) -- (n0_0);
    \draw[->] (obj1.two east) -- (n0_1);
    \draw[->] (obj2.two east) -- (n0_2);
\end{tikzpicture}

    \caption{\texttt{list\_remove(\&l, 1)}}
    \label{fig:sol:list_remove1}
\end{figure}
